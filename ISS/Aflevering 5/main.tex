%%%%%     PACKS     %%%%%
\documentclass[12pt]{article}
\usepackage[margin=1in,headsep=.60in]{geometry}
\usepackage[utf8]{inputenc}
\usepackage[table, dvipsnames]{xcolor}
\usepackage{array}
\usepackage{amsmath}
\usepackage{booktabs}
\usepackage{amssymb}
\usepackage{amsfonts}
\usepackage{siunitx}
\usepackage{graphicx}
\usepackage{caption}
\usepackage{pgfplots}
\graphicspath{{Images/}}
\usepackage[colorinlistoftodos]{todonotes}
\usepackage{cleveref}
\usepackage[labelformat=simple]{subcaption}
\usepackage{grffile}
\usepackage{gensymb}
\usepackage{float}
\usepackage[shortlabels]{enumitem}
\usepackage{enumitem}
\setlistdepth{9}
\usepackage{biblatex} %Imports biblatex package
\addbibresource{references.bib}
\usepackage{outlines}
\usepackage{minted}
\usepackage{pdflscape}
\usepackage{everypage}
\usepackage{multirow}
\usepackage{multicol}
\usepackage{afterpage}
\usepackage[labelfont=bf, font=small]{caption}
\usepackage{upgreek}
\usepackage{tikz}
\usetikzlibrary{automata, positioning}
\usepackage{logicproof}
\usepackage{mdframed}


\usepackage{lastpage}
%%%%%     COMMANDS     %%%%%
\newcommand{\p}{\partial}
\newenvironment{rowequmat}[1]{\left(\array{@{}#1@{}}}{\endarray\right)}


\begin{document}

\begin{titlepage}
\newcommand{\HRule}{\rule{\linewidth}{0.5mm}}
\center

\textsc{\LARGE Aarhus university}\\[1.5cm]
\textsc{\Large Computer-Science}\\[0.5cm]
\textsc{\large Introduction to probability and statistics}\\[0.5cm]
    

\HRule\\[0.4cm]
	
	{\huge\bfseries Handin 5 }\\[0.4cm] % Title of your document
	
\HRule\\[1.5cm]

\begin{minipage}{0.4\textwidth}
		\begin{flushleft}
			\large
			\textit{Author}\\	
			Søren M. \textsc{Damsgaard}\\
                % Your name
		\end{flushleft}
	\end{minipage}
~
	\begin{minipage}{0.4\textwidth}
		\begin{flushright}
			\large
			\textit{Student number}\\
			\textbf{202309814}\\
                % Studienummer\
			
		\end{flushright}
	\end{minipage}

\vfill\vfill\vfill % Position the date 3/4 down the remaining page
	
	{\large\today}

\vfill\vfill
	\includegraphics[width=0.2\textwidth]{Aarhus_University_seal.png}\\[1cm] % Include a department/university logo - this will require the graphicx package
	 

\vfill
\end{titlepage}
%%%%%     CHAPTERS     %%%%%
\hspace{0.02cm}

\section*{Game Strategy with Dice!}
We consider the following strategy:
\[
W = 
\begin{cases}
X, & \text{if } X \ge 4,\\[4pt]
Y, & \text{if } X < 4.
\end{cases}
\]
That is, the strategy consists of stopping the game if the number of eyes (die roll) is 4, 5, or 6, and rolling a new die if the number of eyes is 1, 2, or 3.
\vspace{2cm}
	\begin{mdframed}[backgroundcolor=blue!5,linecolor=blue!20]
	(a) What is the expected payoff of the game if you choose to never roll a new die? That is, you win \(X\) kr.
    \end{mdframed}
	The formula for the expected value from Def 3.11 is:
	\begin{equation}
	    E(X) = \sum_{x_k \in R_X} x_k \cdot P_X(x_k)
	\end{equation}
	For this little case our range is \(R_X = \{1, 2, 3, 4, 5, 6\}\) and the probability is \(P_X(x_k) = \frac{1}{6}\) for all \(x_k \in R_X\) since we are using a die.\\
	This gives us:
	\begin{equation*}
		 E(X) = \sum_{x_k \in R_X} x_k \cdot P_X(x_k) = 1 \cdot \frac{1}{6} + 2 \cdot \frac{1}{6} + 3 \cdot \frac{1}{6} + 4 \cdot \frac{1}{6} + 5 \cdot \frac{1}{6} + 6 \cdot \frac{1}{6}
	\end{equation*}
	This simplifies to:
	\begin{equation*}
	    E(X) = \frac{1 + 2 + 3 + 4 + 5 + 6}{6} = \frac{21}{6} = \underbar{3.5}
	\end{equation*}
\vspace{2cm}
	
	\begin{mdframed}[backgroundcolor=blue!5,linecolor=blue!20]
		(b) Interpret strategy \(W\) in relation to subquestion (a).
	\end{mdframed}
	\textcolor{red}{Strategy (a) takes the first outcome from a single die roll, which just gives a uniform distribution over the die's (dies?) range. Strategy (b) introduces a conditional roll, if the first roll was less than four it rolls again and takes the outcome, this creates a bias towards higher outcomes compared to (a), which should give a higher expected value.}
\vspace{2cm}
\begin{mdframed}[backgroundcolor=blue!5,linecolor=blue!20]
     (c) Find the range \(R_W\) and the PMF \(P_W\) for \(W\). Also, compute the expected payoff when using strategy \(W\).
\end{mdframed}
    Let's take it from the top jesus!\\
	The range is the same as before in (a):
	\[
	R_W = \{1, 2, 3, 4, 5, 6\}
	\]
	now for the PMF.\\
	The probability of getting a 1, 2 or 3 is rolling 1, 2 or 3 twice in a row.\\
	The first roll is \(X\) and the second roll is \(Y\) and they are independent events.\\
	\begin{equation*}
	    P_W(w) = P(X < 4) \cdot P(Y = w) = \frac{1}{2} \cdot \frac{1}{6} = \frac{1}{12} \;\;\;\text{ for } w = 1, 2, 3
	\end{equation*}
	
	And the probability of getting a 4, 5 or 6 is either getting them on the first role or getting a 1, 2 or 3 and then rolling a 4, 5 or 6.\\
	\textcolor{red}{Note that the events $P(X = w) + P(X < 4)$ are disjoint so we can add the probabilities of multiple events using the 'Inclusion-Exclusion Principle' since their intersection is always zero according to chapter 1.2.2~\cite{STAT}.}
	\begin{equation*}
	    P_W(w) = P(X = w) + P(X < 4) \cdot P(Y = w) =  \frac{1}{6} + \frac{1}{2} \cdot \frac{1}{6} = \frac{1}{4} \;\;\;\text{ for } w = 4, 5, 6
	\end{equation*}
	
	\begin{equation}
		P_W(W = w) =  
	\begin{cases}
	\frac{1}{12}, & \text{if } w = 1\\[4pt]
	\frac{1}{12}, & \text{if } w = 2\\[4pt]
	\frac{1}{12}, & \text{if } w = 3\\[4pt]
	\frac{1}{4}, & \text{if } w = 4\\[4pt]
	\frac{1}{4}, & \text{if } w = 5\\[4pt]
	\frac{1}{4}, & \text{if } w = 6
	\end{cases}
	\end{equation}
	We can do a sanity check with \textcolor{red}{axiom 2 of probability from chapter 1.3.2~\cite{STAT} to make sure our PMF is correct:}
	\begin{equation*}
	    \sum_{w_k \in R_W} P_W(w_k) = \frac{1}{12} + \frac{1}{12} + \frac{1}{12} + \frac{1}{4} + \frac{1}{4} + \frac{1}{4} = 1
	\end{equation*}
	Now for the expected value of \(W\):\\
	\begin{equation*}
	    E(W) = \sum_{w_k \in R_W} w_k \cdot P_W(w_k) = 1 \cdot \frac{1}{12} + 2 \cdot \frac{1}{12} + 3 \cdot \frac{1}{12} + 4 \cdot \frac{1}{4} + 5 \cdot \frac{1}{4} + 6 \cdot \frac{1}{4}
	\end{equation*}
	This simplifies to:
	\begin{equation*}
	    E(W) = \frac{1 + 2 + 3}{12} + \frac{4 + 5 + 6}{4} = \frac{6}{12} + \frac{15}{4} = \frac{51}{12} = \underbar{4.25}
	\end{equation*}
	Which gives us a higher expected value than strategy (a) as expected.
	\[
	\underbar{E(W) = 4.25 $>$ E(X) = 3.5}
	\]
\end{document}