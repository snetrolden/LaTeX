%%%%%     PACKS     %%%%%
\documentclass[12pt]{article}
\usepackage[margin=1in,headsep=.60in]{geometry}
\usepackage[utf8]{inputenc}
\usepackage[table, dvipsnames]{xcolor}
\usepackage{array}
\usepackage{amsmath}
\usepackage{booktabs}

\usepackage{amssymb}
\usepackage{amsfonts}
\usepackage{siunitx}
\usepackage{graphicx}
\usepackage{caption}
\usepackage{pgfplots}
\graphicspath{{Images/}}
\usepackage[colorinlistoftodos]{todonotes}
\usepackage{cleveref}
\usepackage[labelformat=simple]{subcaption}
\usepackage{grffile}
\usepackage{gensymb}
\usepackage{float}
\usepackage[shortlabels]{enumitem}
\usepackage{enumitem}
\setlistdepth{9}
\usepackage{biblatex} %Imports biblatex package
\addbibresource{references.bib}
\usepackage{outlines}
\usepackage{minted}
\usepackage{pdflscape}
\usepackage{everypage}
\usepackage{multirow}
\usepackage{multicol}
\usepackage{afterpage}
\usepackage[labelfont=bf, font=small]{caption}
\usepackage{upgreek}
\usepackage{tikz}
\usetikzlibrary{automata, positioning}
\usepackage{logicproof}


\usepackage{lastpage}
%%%%%     COMMANDS     %%%%%
\newcommand{\p}{\partial}
\newenvironment{rowequmat}[1]{\left(\array{@{}#1@{}}}{\endarray\right)}


\begin{document}

\begin{titlepage}
\newcommand{\HRule}{\rule{\linewidth}{0.5mm}}
\center

\textsc{\LARGE Aarhus university}\\[1.5cm]
\textsc{\Large Computer-Science}\\[0.5cm]
\textsc{\large Introduction to probability and statistics}\\[0.5cm]
    

\HRule\\[0.4cm]
	
	{\huge\bfseries Handin 4 }\\[0.4cm] % Title of your document
	
\HRule\\[1.5cm]

\begin{minipage}{0.4\textwidth}
		\begin{flushleft}
			\large
			\textit{Author}\\	
			Søren M. \textsc{Damsgaard}\\
                % Your name
		\end{flushleft}
	\end{minipage}
~
	\begin{minipage}{0.4\textwidth}
		\begin{flushright}
			\large
			\textit{Student number}\\
			\textbf{202309814}\\
                % Studienummer\
			
		\end{flushright}
	\end{minipage}

\vfill\vfill\vfill % Position the date 3/4 down the remaining page
	
	{\large\today}

\vfill\vfill
	\includegraphics[width=0.2\textwidth]{Aarhus_University_seal.png}\\[1cm] % Include a department/university logo - this will require the graphicx package
	 

\vfill
\end{titlepage}
%%%%%     CHAPTERS     %%%%%
\hspace{0.02cm}

\section*{Let $0 < p < 1 $ and $X \approx $ Geometric($p$).}

\subsection*{Calculate The cumulative distribution function (CDF) $F_X(x) = P(X \leq x)$ for all $x = 1,2,3,.....$}

\subsection*{Show that $X$ has the 'lack of memory' property, i.e.}
\begin{equation}
P(X > m + k | X > m) = P(X > k) \text{ for all } k,m = 1,2,3,....
\end{equation}
\\
\\
\subsection*{Finding the CDF (Dora's solution)}
The CDF is essentially the sum of all the probabilities up to and including $x$ for discrete variables. \\
This means that we can express it as:
\begin{equation}
F_X(x) = P(X \leq x) = \sum_{i=1}^{x} P(X = i) \;\; \text{ for } x = 1,2,3,...
\end{equation}
From Def 3.5 \cite{STAT} we know that the PMF of a geometric distribution is:
\begin{equation}
P(X = i) = (1-p)^{i-1}p
\end{equation}
This means that we can rewrite the CDF as:
\begin{equation}
F_X(x) = \sum_{i=1}^{x} (1-p)^{i-1}p
\end{equation}
Move the constant out of the summation:
\begin{equation}
F_X(x) = p \sum_{i=1}^{x} (1-p)^{i-1}
\end{equation}
We will now show that the current summation is a geometric series.\\
A geometric series is defined as:
\begin{equation}
a + ax + ax^2 + ax^3 + ... + ax^{n-1} = \sum_{k=0}^{n-1} ax^k = a \frac{1-x^n}{1-x}
\end{equation}
from example 1.12 equation 1.3 \cite{STAT}.\\
We can see that our summation is a geometric series with $a = 1$, $x = (1-p)$ and $n = x$. This means that we can rewrite the CDF as:
\begin{equation}
F_X(x) = p \cdot \frac{1-(1-p)^x}{1-(1-p)}
\end{equation}
which simplifies to:
\begin{equation}
F_X(x) = 1-(1-p)^x
\end{equation}
\\\\
\boxed{\text{Note that it's important when we use the geometric series we, it has to start at indexx 0, so we had to rewrite the summation to start at 0 instead of 1 in the future\\
this case is a special case where it works, but in general it's bad practice.}}

\subsection*{Show X has dementia (lack of memory property)}
Reiterating equation(1)
\[
P(X > m + k | X > m) = P(X > k) \text{ for all } k,m = 1,2,3,....
\]
We use the definition of conditional probability from chapter 1.4 \cite{STAT} to do legal crime:
\begin{equation}
P(A|B) = \frac{P(A \cap B)}{P(B)}
\end{equation}
We can then rewrite the right side of equation(1) as:
\begin{equation}
P(X > m + k | X > m) = \frac{P((X > m + k) \cap (X > m))}{P(X > m)} = \frac{P(X > m + k)}{P(X > m)}
\end{equation}
We note here that $P((X > m + k) \cap (X > m)) = P(X > m + k)$ since $X > m + k \implies X > m$.\\\\
We can now use the CDF we found earlier to express $P(X > x)$ using the complement, like from example 3.10 \cite{STAT}:
\begin{equation}
P(X > x) = 1 - P(X \leq x) = 1 - F_X(x)
\end{equation}
We can then take equation(10) and throw our new math at it like Kobe Bryant on January 22, 2006 (60 points on 50 shots):
\begin{equation}
P(X > m + k | X > m) = \frac{1 - F_X(m + k)}{1 - F_X(m)}
\end{equation}
We can then insert the value of the CDF:
\begin{equation}
P(X > m + k | X > m) = \frac{1 - (1-(1-p)^{m+k})}{1 - (1-(1-p)^m)}
\end{equation}
which we can beautify a little bit by removing the negative vibes:
\begin{equation}
P(X > m + k | X > m) = \frac{(1-p)^{m+k}}{(1-p)^m}
\end{equation}
And as a finishing touch we will put the crown on it by simplifying the fraction even further:
\begin{equation}
P(X > m + k | X > m) = (1-p)^k
\end{equation}
Now we can see that the right side of equation(1) is:
\begin{equation}
P(X > k) = 1 - F_X(k) = 1 - (1-(1-p)^k) = (1-p)^k
\end{equation}
\printbibliography % This prints the bibliography

\end{document}