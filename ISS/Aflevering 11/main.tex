%%%%%     PACKS     %%%%%
\documentclass[12pt]{article}
\usepackage[margin=1in,headsep=.60in]{geometry}
\usepackage[utf8]{inputenc}
\usepackage[table, dvipsnames]{xcolor}
\usepackage{array}
\usepackage{amsmath}
\usepackage{booktabs}
\usepackage{mdframed} %For box around text
\usepackage{hyperref} %For hyperlinks in the PDF
\hypersetup{
	colorlinks=true,
	linkcolor=blue,
	filecolor=magenta,
	urlcolor=cyan,
}
\urlstyle{same}
\usepackage{amsthm}
\usepackage{amssymb}
\usepackage{amsfonts}
\usepackage{siunitx}
\usepackage{graphicx}
\usepackage{caption}
\usepackage{pgfplots}
\graphicspath{{Images/}}
\usepackage[colorinlistoftodos]{todonotes}
\usepackage{cleveref}
\usepackage[labelformat=simple]{subcaption}
\usepackage{grffile}
\usepackage{gensymb}
\usepackage{float}
\usepackage[shortlabels]{enumitem}
\usepackage{enumitem}
\setlistdepth{9}
\usepackage{biblatex} %Imports biblatex package
\addbibresource{references.bib}
\usepackage{outlines}
\usepackage{minted}
\usepackage{pdflscape}
\usepackage{everypage}
\usepackage{multirow}
\usepackage{multicol}
\usepackage{afterpage}
\usepackage[labelfont=bf, font=small]{caption}
\usepackage{upgreek}
\usepackage{tikz}
\usetikzlibrary{automata, positioning}
\usepackage{logicproof}


\usepackage{lastpage}
%%%%%     COMMANDS     %%%%%
%%%% Blue box for subsection text (not figures) %%%%
\newenvironment{bluebox}
  {\begin{mdframed}[backgroundcolor=blue!5,linecolor=blue!40,roundcorner=8pt]}
  {\end{mdframed}}

%%%% Simple neutral box for figures %%%%
\newenvironment{figbox}
  {\begin{mdframed}[roundcorner=8pt,shadow=true,shadowsize=4pt,shadowcolor=black!40]}
  {\end{mdframed}}




\begin{document}

\begin{titlepage}
\newcommand{\HRule}{\rule{\linewidth}{0.5mm}}




\center

\textsc{\LARGE Aarhus university}\\[1.5cm]
\textsc{\Large Computer-Science}\\[0.5cm]
\textsc{\large Introduction to probability and statistics}\\[0.5cm]
    

\HRule\\[0.4cm]
	\center	
	{\huge\bfseries Handin 11 }\\[0.4cm] % Title of your document
\HRule\\[1.5cm]

\begin{minipage}{0.4\textwidth}
		\begin{flushleft}
			\large
			\textit{Author}\\	
			Søren M. \textsc{Damsgaard}\\
                % Your name
		\end{flushleft}
	\end{minipage}
~
	\begin{minipage}{0.4\textwidth}
		\begin{flushright}
			\large
			\textit{Student number}\\
			\textbf{202309814}\\
                % Studienummer\
			
		\end{flushright}
	\end{minipage}

\vfill\vfill\vfill % Position the date 3/4 down the remaining page
	
	{\large\today}

\vfill\vfill
	\includegraphics[width=0.2\textwidth]{Aarhus_University_seal.png}\\[1cm] % Include a department/university logo - this will require the graphicx package
	 

\vfill
\end{titlepage}
%%%%%     CHAPTERS     %%%%%
\hspace{0.02cm}

\begin{bluebox}
In this exercise, we assume that the lifetime (measured in years) of phones follows an exponential distribution, and that we have observed the lifetimes of 10 phones.  
We are interested in testing the null hypothesis that the mean lifetime $\mu$ of phones is 2 years, against the alternative hypothesis that the mean lifetime is strictly less than 2 years.\\
\textit{Hint:} Use Theorem B from weekly sheet 8.
\end{bluebox}
\subsubsection*{
Formulate the model described above and the corresponding hypotheses mathematically, and relate $\mu$ to the parameter of the exponential distribution.  
}
We have 10 independant observations $X_1, X_2, \ldots, X_{10}$ from an exponential distribution with parameter $\lambda > 0$. The probability density function of an exponential distribution is given by
\[
f(x; \lambda) = \lambda e^{-\lambda x}, \quad x > 0.
\]
The mean of an exponential distribution is given by $\mu = \frac{1}{\lambda}$\\
With this we set up our hypotheses:
\[H_0: \mu = 2 \quad \text{vs} \quad H_1: \mu < 2\]
or in terms of $\lambda$: (Please make a note if intermediate steps are needed)
\[H_0: \lambda = \frac{1}{2} \quad \text{vs} \quad H_1: \lambda > \frac{1}{2}\]
\begin{bluebox}
Let $W$ denote the sample mean of our 10 observations, and let $c > 0$ be a positive number. We will now construct an exact (i.e., non-asymptotic) test for the null hypothesis as follows:  
Accept the null hypothesis if $W \geq c$, and otherwise reject it.
\end{bluebox}
\subsubsection*{
Find the value of $c$ such that the test has a significance level of $\alpha = 0.05$. 
}
Re-iterating what the problem states, we have the test:
\[\text{Accept } H_0 \text{ if } W \geq c, \text{ else reject } H_0\]
For this we need to find $W$ to isolate $c$ and express it in a way we can work with.\\
We start by finding the distribution of $W$.\\
By definition we define $W = \bar{X}$:
\[
W = \frac{1}{n}\sum_{i=1}^{n} X_i = \frac{1}{10}\sum_{i=1}^{10} X_i
\]
And since it's a sum of exponential dsitribution, we can use Theorem B from 'Ugeseddel 8', which essentially states:
\[
X \sim \mathrm{exponential}(\lambda) \implies \sum_{i=1}^{n} X_i \sim \mathrm{gamma}(n, \lambda)
\]
We can use that to further define $W$
\[
W = \frac{1}{10} S \quad \text{for } S \sim \mathrm{gamma}(10, \lambda)
\]
Here it's important to remember the rules for scaling a gamma distribution, which leaves us with:
\[
W \sim \mathrm{gamma}(10, 10 \lambda)
\]
Now, for a significance level of $\alpha = 0.05$ we have:
\[P(\text{Reject } H_0 | H_0 \text{ true}) = P(W < c | \lambda = \frac{1}{2}) = 0.05
\]
Which reminds us of the definition of the CDF $P(X \leq x)$, but it can also be written as:
\[
F_X(x) = P(X < x) - P(X = x) 
\]
since $P(X = x) = 0$ for continuous distributions, we have:
\[
F_W(c) = P(W < c) = 0.05
\]
With this we can find $c$ using the inverse CDF in R using the command:
\[
\text{qgamma(0.05, shape = 10, rate = 5)}
\]
Which gives us:
\[c \approx 1.085\]
This gives us a more complete test:
\[\text{Accept } H_0 \text{ if } W \geq 1.085, \text{ else reject } H_0\]

\subsubsection*{
Now assume that we have observed the lifetimes of 1000 phones. Find the value of $c$ such that the test again has a significance level of $\alpha = 0.05$, and compare the result with the one obtained in previous question
}
This part here is interesting in that we can use the same approach as before but with different parameters.\\
So we have:
\[
\text{qgamma(0.05, shape = 1000, rate = 500)}
\]
Which gives us:
\[
c \approx 1.897
\]
Something important to note here is that, per the 'Weak Law of Large Nummbers' our sample mean will converge to the true mean as $n$ approaches infinity.\\
So with a larger $n$ we would see and expect $c$ to be closer to the true mean of 2, which is the trend we see when comparing the two c values. 
\printbibliography% This prints the bibliography
\end{document}