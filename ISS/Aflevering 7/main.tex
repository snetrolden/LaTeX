%%%%%     PACKS     %%%%%
\documentclass[12pt]{article}
\usepackage[margin=1in,headsep=.60in]{geometry}
\usepackage[utf8]{inputenc}
\usepackage[table, dvipsnames]{xcolor}
\usepackage{array}
\usepackage{amsmath}
\usepackage{booktabs}
\usepackage{mdframed} %For box around text

\usepackage{amssymb}
\usepackage{amsfonts}
\usepackage{siunitx}
\usepackage{graphicx}
\usepackage{caption}
\usepackage{pgfplots}
\graphicspath{{Images/}}
\usepackage[colorinlistoftodos]{todonotes}
\usepackage{cleveref}
\usepackage[labelformat=simple]{subcaption}
\usepackage{grffile}
\usepackage{gensymb}
\usepackage{float}
\usepackage[shortlabels]{enumitem}
\usepackage{enumitem}
\setlistdepth{9}
\usepackage{biblatex} %Imports biblatex package
\addbibresource{references.bib}
\usepackage{outlines}
\usepackage{minted}
\usepackage{pdflscape}
\usepackage{everypage}
\usepackage{multirow}
\usepackage{multicol}
\usepackage{afterpage}
\usepackage[labelfont=bf, font=small]{caption}
\usepackage{upgreek}
\usepackage{tikz}
\usetikzlibrary{automata, positioning}
\usepackage{logicproof}


\usepackage{lastpage}
%%%%%     COMMANDS     %%%%%
%%%% Blue box for subsection text (not figures) %%%%
\newenvironment{bluebox}
  {\begin{mdframed}[backgroundcolor=blue!5,linecolor=blue!40,roundcorner=8pt]}
  {\end{mdframed}}

%%%% Simple neutral box for figures %%%%
\newenvironment{figbox}
  {\begin{mdframed}[roundcorner=8pt,shadow=true,shadowsize=4pt,shadowcolor=black!40]}
  {\end{mdframed}}




\begin{document}

\begin{titlepage}
\newcommand{\HRule}{\rule{\linewidth}{0.5mm}}




\center

\textsc{\LARGE Aarhus university}\\[1.5cm]
\textsc{\Large Computer-Science}\\[0.5cm]
\textsc{\large Introduction to probability and statistics}\\[0.5cm]
    

\HRule\\[0.4cm]
	\center	
	{\huge\bfseries Handin 7 }\\[0.4cm] % Title of your document
\HRule\\[1.5cm]

\begin{minipage}{0.4\textwidth}
		\begin{flushleft}
			\large
			\textit{Author}\\	
			Søren M. \textsc{Damsgaard}\\
                % Your name
		\end{flushleft}
	\end{minipage}
~
	\begin{minipage}{0.4\textwidth}
		\begin{flushright}
			\large
			\textit{Student number}\\
			\textbf{202309814}\\
                % Studienummer\
			
		\end{flushright}
	\end{minipage}

\vfill\vfill\vfill % Position the date 3/4 down the remaining page
	
	{\large\today}

\vfill\vfill
	\includegraphics[width=0.2\textwidth]{Aarhus_University_seal.png}\\[1cm] % Include a department/university logo - this will require the graphicx package
	 

\vfill
\end{titlepage}
%%%%%     CHAPTERS     %%%%%
\hspace{0.02cm}
\begin{bluebox}
\subsection*{Let $X$ be a continuous random variable with the probability density functuion (PDF) $f_X$}
$f_X$ is given by:
\[
    f_X(x) =
    \begin{cases}
        3x^2 & \text{for } 0<x<1,\\
        0 & \text{otherwise}.
    \end{cases}
\]
The set $Y = -\frac{1}{2}\log(X)$. Find the PDF $f_Y$ for $Y$. As always, indicate the distribution of $Y$ if it is a known distribution.
\end{bluebox}
For this we shal utilize Theorem 4.1' from 'Uge Seddel 6'
\begin{figbox}
\begin{figure}[H]
	\centering
	\includegraphics[width=1\textwidth]{Theorem41.png}
	\label{fig:Theorem 4.1}
\end{figure}
\end{figbox}
We start by defining a function $g(x) = -\frac{1}{2}\log(x)$, then let $Y = g(X)$.\\
Now we show its monotonic and differentiable by taking the derivative of $g$:
\[
	g'(x) = -\frac{1}{2} \cdot \frac{1}{x} = -\frac{1}{2x}		
\]
We can see that $g$ is monotonic (strongly decreasing) and differentiable on the interval $(0,1)$, our range $R_X$, so we can use Theorem 4.1'.
\subsubsection*{The Inverse}
We define the inverse function $g^{-1}(y) = h(y)$: 
\[
	h(y) = e^{-2y}
\] 
we can check this by plugging $h$ into $g$, it's important to note the fact that $\log(e) = 1$ and $e^{\log(y)} = y$:
\[
	g(h(y)) = -\frac{1}{2}\log(e^{-2y}) = -\frac{1}{2} \cdot -2y \cdot \log(e) = -\frac{1}{2} \cdot -2y = y
\]
And then plugging $g$ into $h$:
\[
	h(g(y)) = e^{-2 \cdot (-\frac{1}{2})\log(y)} = e^{\log(y)} = y
\]
Since they both equal $y$, we can say that they are inverses (Source? WebMatematik).
\subsubsection*{The derivative of the Inverse}
Let's plug it in:
\[
h'(y) = \frac{d}{dy}e^{-2y} = -2e^{-2y}
\]

\subsubsection*{The PDF of $X$ at $h(y)$?}
That is just $f_X(h(y))$ where we have the PDF of $f_X(x) = 3x^2$:
\[
	f_X(h(y)) = f_X(e^{-2y}) = 3(e^{-2y})^2 = 3e^{-4y}
\]
For this we used a small trick of $(a^b)^c = a^b \cdot a^b \cdot a^b \ldots = a^{b+b+\cdots+b} = a^{b \cdot c}$ where $c$ is the number of times we multiply $a^b$ with itself (more of a note to myself OwO).
\subsubsection*{PDF of $Y$ + Conclusion}
Now we have all the parts we need to find $f_Y(y)$:
\[
	f_Y(y) = |h'(y)| \cdot f_X(h(y)) = |-2e^{-2y}| \cdot 3e^{-4y} = 6e^{-6y}
\]
Now we need to find the range $R_Y$ of $Y$. We can do this by finding the minimum and maximum of $Y$:
\[	
	\min(Y) = g(1) = -\frac{1}{2}\log(1) = 0
\]
\[
	\max(Y) = g(0) = -\frac{1}{2}\log(0) = \infty
\]
So we have $R_Y = (0, \infty)$.\\
So now we have the final PDF of $Y$:
\[
	f_Y(y) =
	\begin{cases}
		6e^{-6y} & \text{for } 0<y<\infty,\\
		0 & \text{otherwise}.
	\end{cases}	
\]
This PDF has the form of an exponential distribution with parameter $\lambda = 6$\\
 or $Y \sim \text{Exponential}(6)$
\[
 f_X(x) = \lambda e^{-\lambda x} \text{ for } x > 0 \; \;  \leftarrow \textit{ PDF for an exponential distribution}
\]
Thanks for reading my TED-talk on \textbf{Math fueled by coffee and need for money!}\\

\printbibliography% This prints the bibliography

\end{document}