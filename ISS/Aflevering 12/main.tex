%%%%%     PACKS     %%%%%
\documentclass[12pt]{article}
\usepackage[margin=1in,headsep=.60in]{geometry}
\usepackage[utf8]{inputenc}
\usepackage[table, dvipsnames]{xcolor}
\usepackage{array}
\usepackage{amsmath}
\usepackage{booktabs}
\usepackage{mdframed} %For box around text
\usepackage{hyperref} %For hyperlinks in the PDF
\hypersetup{
	colorlinks=true,
	linkcolor=blue,
	filecolor=magenta,
	urlcolor=cyan,
}
\urlstyle{same}
\usepackage{amsthm}
\usepackage{amssymb}
\usepackage{amsfonts}
\usepackage{siunitx}
\usepackage{graphicx}
\usepackage{caption}
\usepackage{pgfplots}
\graphicspath{{Images/}}
\usepackage[colorinlistoftodos]{todonotes}
\usepackage{cleveref}
\usepackage[labelformat=simple]{subcaption}
\usepackage{grffile}
\usepackage{gensymb}
\usepackage{float}
\usepackage[shortlabels]{enumitem}
\usepackage{enumitem}
\setlistdepth{9}
\usepackage{biblatex} %Imports biblatex package
\addbibresource{references.bib}
\usepackage{outlines}
\usepackage{minted}
\usepackage{pdflscape}
\usepackage{everypage}
\usepackage{multirow}
\usepackage{multicol}
\usepackage{afterpage}
\usepackage[labelfont=bf, font=small]{caption}
\usepackage{upgreek}
\usepackage{tikz}
\usetikzlibrary{automata, positioning}
\usepackage{logicproof}


\usepackage{lastpage}
%%%%%     COMMANDS     %%%%%
%%%% Blue box for subsection text (not figures) %%%%
\newenvironment{bluebox}
  {\begin{mdframed}[backgroundcolor=blue!5,linecolor=blue!40,roundcorner=8pt]}
  {\end{mdframed}}

%%%% Simple neutral box for figures %%%%
\newenvironment{figbox}
  {\begin{mdframed}[roundcorner=8pt,shadow=true,shadowsize=4pt,shadowcolor=black!40]}
  {\end{mdframed}}




\begin{document}

\begin{titlepage}
\newcommand{\HRule}{\rule{\linewidth}{0.5mm}}

\center

\textsc{\LARGE Aarhus university}\\[1.5cm]
\textsc{\Large Computer-Science}\\[0.5cm]
\textsc{\large Introduction to probability and statistics}\\[0.5cm]
    

\HRule\\[0.4cm]
    \center 
    {\huge\bfseries Handin 12 }\\[0.4cm] % Title of your document
\HRule\\[1.5cm]

\begin{minipage}{0.4\textwidth}
        \begin{flushleft}
            \large
            \textit{Author}\\   
            Søren M. \textsc{Damsgaard}\\
                 % Your name
        \end{flushleft}
    \end{minipage}
~
    \begin{minipage}{0.4\textwidth}
        \begin{flushright}
            \large
            \textit{Student number}\\
            \textbf{202309814}\\
                 % Studienummer\
            
        \end{flushright}
    \end{minipage}

\vfill\vfill\vfill % Position the date 3/4 down the remaining page
    
    {\large\today}

\vfill\vfill
    \includegraphics[width=0.2\textwidth]{Aarhus_University_seal.png}\\[1cm] % Include a department/university logo - this will require the graphicx package
    

\vfill
\end{titlepage}
%%%%%      CHAPTERS      %%%%%
\hspace{0.02cm}

\section*{Last Assignment}

\begin{bluebox}
We want to test the null hypothesis that 'at least 10\% of the students suffer from allergies'. We have collected data from 225 students, where 21 of them suffer from allergies. Consider the following statistical model for our data: Let $X_1, \ldots, X_{225}$ denote the sample, where $X_i \sim \mathrm{Bernoulli}(\theta)$ and $\theta \in (0, 1)$ is an unknown parameter. Consider the following hypotheses:
$$
H_0 : \theta \geq 0.1 \quad \text{vs} \quad H_1 : \theta < 0.1
$$
\end{bluebox}

---

\subsubsection*{
(a) Argue that the above model corresponds to a sample with unknown mean and variance, and likewise argue that $H_0$ corresponds to the statement that at least 10\% of the students suffer from allergies.
}
\textit{Solution for (a)}:
\begin{enumerate}[i.]
    \item **Unknown Mean and Variance:** For a Bernoulli distribution $X_i \sim \mathrm{Bernoulli}(\theta)$, the mean is $E[X_i] = \theta$ and the variance is $\mathrm{Var}[X_i] = \theta(1-\theta)$. Since $\theta$ is an **unknown parameter**, both the mean and the variance of the underlying distribution are unknown.
    \item **$H_0$ Interpretation:** $\theta$ represents the true proportion (or probability) of students in the population who suffer from allergies. The null hypothesis $H_0: \theta \geq 0.1$ states that this true proportion is greater than or equal to $0.1$ (or $10\%$). This directly translates to the statement that **at least 10\% of the students suffer from allergies**.
\end{enumerate}

---

\subsubsection*{
(b) Perform an asymptotic hypothesis test with a significance level $\alpha = 0.05$ for the null hypothesis.
}
\textit{Solution for (b)}:
We perform a Z-test for a population proportion, $\theta$.
\begin{enumerate}[i.]
    \item **Hypotheses and Significance Level:**
    $$H_0 : \theta \geq 0.1 \quad \text{vs} \quad H_1 : \theta < 0.1 \quad (\text{Left-tailed test})$$
    $$\alpha = 0.05$$
    \item **Observed Sample Proportion ($\hat{\theta}$):**
    The number of successes is $Y = 21$, and the sample size is $n = 225$.
    $$\hat{\theta} = \frac{Y}{n} = \frac{21}{225} \approx 0.0933$$
    \item **Test Statistic ($Z$):**
    The asymptotic test statistic uses the value under the null hypothesis, $\theta_0 = 0.1$.
    $$Z = \frac{\hat{\theta} - \theta_0}{\sqrt{\frac{\theta_0(1-\theta_0)}{n}}}$$
    $$Z = \frac{0.0933 - 0.1}{\sqrt{\frac{0.1(1-0.1)}{225}}} = \frac{-0.0067}{\sqrt{\frac{0.09}{225}}} = \frac{-0.0067}{\sqrt{0.0004}} = \frac{-0.0067}{0.02}$$
    $$Z \approx -0.335$$
    \item **Critical Value ($Z_{\alpha}$):**
    Since this is a left-tailed test with $\alpha = 0.05$, the critical value $Z_{0.05}$ is found such that $P(Z \leq Z_{0.05}) = 0.05$.
    $$Z_{0.05} \approx -1.645$$
    \item **Conclusion:**
    We compare the calculated test statistic ($Z \approx -0.335$) with the critical value ($Z_{0.05} \approx -1.645$).
    Since $Z > Z_{0.05}$ (i.e., $-0.335 > -1.645$), the test statistic does not fall into the rejection region. We **do not reject $H_0$**.
    \textit{Conclusion in context:} There is not enough evidence at the $5\%$ significance level to conclude that the true proportion of students suffering from allergies is less than 10\%.
\end{enumerate}

---

\subsubsection*{
(c) Calculate the corresponding P-value.
}
\textit{Solution for (c)}:
The P-value is the probability of observing a result as extreme as, or more extreme than, the one observed, assuming $H_0$ is true. Since this is a left-tailed test, we calculate $P(Z \leq Z_{\text{obs}})$ where $Z_{\text{obs}} \approx -0.335$:
$$
\text{P-value} = P(Z \leq -0.335)
$$
Using the standard normal distribution (Z-table or calculator):
$$
\text{P-value} \approx 0.3687
$$
Since the P-value ($0.3687$) is much larger than the significance level ($\alpha = 0.05$), we confirm the decision from part (b) to **not reject the null hypothesis $H_0$**.
\\

\[
f_{X,Y}(x,y) = \frac{2}{\sqrt{\pi}}e^{-x^2-2y} \; \text{for } y > 0
\]
find distribution for X and say if X and Y are independent?
\[
f_{X}(x) = \int_{0}^{\infty} f_{X,Y}(x,y) dy = \int_{0}^{\infty} \frac{2}{\sqrt{\pi}}e^{-x^2-2y} dy = \frac{2}{\sqrt{\pi}}e^{-x^2} \int_{0}^{\infty} e^{-2y} dy = \frac{2}{\sqrt{\pi}}e^{-x^2} \cdot \frac{1}{2} = \frac{1      }{\sqrt{\pi}}e^{-x^2}
\]
Now for Y:
\[
f_{Y}(y) = \int_{-\infty}^{\infty} f_{X,Y}(x,y) dx = \int_{-\infty}^{\infty} \frac{2}{\sqrt{\pi}}e^{-x^2-2y} dx = \frac{2}{\sqrt{\pi}}e^{-2y} \int_{-\infty}^{\infty} e^{-x^2} dx = \frac{2}{\sqrt{\pi}}e^{-2y} \cdot \sqrt{\pi} = 2e^{-2y}
\]
To check for independence, we see if \( f_{X,Y}(x,y) = f_{X}(x) \cdot f_{Y}(y) \):
\[
f_{X}(x) \cdot f_{Y}(y) = \left(\frac{1}{\sqrt{\pi}}e^{-x^2}\right) \cdot   \left(2e^{-2y}\right) = \frac{2}{\sqrt{\pi}}e^{-x^2-2y} = f_{X,Y}(x,y)
\]
Thus, X and Y are independent.\\

Find the distribution of $Z = e^{\frac{Y}{3}}-1$.
To find the distribution of \( Z = e^{\frac{Y}{3}} - 1 \), we first need to find the cumulative distribution function (CDF) of \( Z \).
\[F_{Z}(z) = P(Z \leq z) = P\left(e^{\frac{Y}{3}} - 1 \leq z\right) = P\left(Y \leq 3\ln(z + 1)\right)\]
Now, we need to find the CDF of \( Y \):
\[F_{Y}(y) = \int_{0}^{y} f_{Y}(t) dt = \int_{0}^{y} 2e^{-2t} dt = \left[-e^{-2t}\right]_{0}^{y} = 1 - e^{-2y}\]
Substituting back into the CDF of \( Z \):
\[F_{Z}(z) = F_{Y}(3\ln(z + 1)) = 1 - e^{-2 \cdot 3\ln(z + 1)} = 1 - e^{-6\ln(z + 1)} = 1 - (z + 1)^{-6}\]
To find the probability density function (PDF) of \( Z \), we differentiate the CDF:
\[f_{Z}(z) = \frac{d}{dz} F_{Z}(z) = \frac{d}{dz} \left(1 - (z + 1)^{-6}\right) = 6(z + 1)^{-7}\]
Thus, the distribution of \( Z \) is given by the PDF:
\[f_{Z}(z) = 6(z + 1)^{-7} \; \text{for } z > -1\]    




\printbibliography
\end{document}
