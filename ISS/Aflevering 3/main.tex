%%%%%     PACKS     %%%%%
\documentclass[12pt]{article}
\usepackage[margin=1in,headsep=.60in]{geometry}
\usepackage[utf8]{inputenc}
\usepackage[table, dvipsnames]{xcolor}
\usepackage{array}
\usepackage{amsmath}
\usepackage{booktabs}

\usepackage{amssymb}
\usepackage{amsfonts}
\usepackage{siunitx}
\usepackage{graphicx}
\usepackage{caption}
\usepackage{pgfplots}
\graphicspath{{Images/}}
\usepackage[colorinlistoftodos]{todonotes}
\usepackage{cleveref}
\usepackage[labelformat=simple]{subcaption}
\usepackage{grffile}
\usepackage{gensymb}
\usepackage{float}
\usepackage[shortlabels]{enumitem}
\usepackage{enumitem}
\setlistdepth{9}
\usepackage{biblatex} %Imports biblatex package
\addbibresource{references.bib}
\usepackage{outlines}
\usepackage{minted}
\usepackage{pdflscape}
\usepackage{everypage}
\usepackage{multirow}
\usepackage{multicol}
\usepackage{afterpage}
\usepackage[labelfont=bf, font=small]{caption}
\usepackage{upgreek}
\usepackage{tikz}
\usetikzlibrary{automata, positioning}
\usepackage{logicproof}


\usepackage{lastpage}
%%%%%     COMMANDS     %%%%%
\newcommand{\p}{\partial}
\newenvironment{rowequmat}[1]{\left(\array{@{}#1@{}}}{\endarray\right)}


\begin{document}

\begin{titlepage}
\newcommand{\HRule}{\rule{\linewidth}{0.5mm}}
\center

\textsc{\LARGE Aarhus university}\\[1.5cm]
\textsc{\Large Computer-Science}\\[0.5cm]
\textsc{\large Introduction to probability and statistics}\\[0.5cm]
    

\HRule\\[0.4cm]
	
	{\huge\bfseries Handin 3 }\\[0.4cm] % Title of your document
	
\HRule\\[1.5cm]

\begin{minipage}{0.4\textwidth}
		\begin{flushleft}
			\large
			\textit{Author}\\	
			Søren M. \textsc{Damsgaard}\\
                % Your name
		\end{flushleft}
	\end{minipage}
~
	\begin{minipage}{0.4\textwidth}
		\begin{flushright}
			\large
			\textit{Student number}\\
			\textbf{202309814}\\
                % Studienummer\
			
		\end{flushright}
	\end{minipage}

\vfill\vfill\vfill % Position the date 3/4 down the remaining page
	
	{\large\today}

\vfill\vfill
	\includegraphics[width=0.2\textwidth]{Aarhus_University_seal.png}\\[1cm] % Include a department/university logo - this will require the graphicx package
	 

\vfill
\end{titlepage}
%%%%%     CHAPTERS     %%%%%
\hspace{0.02cm}

\section*{Problem 7 (Exam Winter 2017/2018)}

It is given that $10\%$ of the population has influenza in a given period.  

\begin{enumerate}
    \item Suppose that $20$ people are chosen randomly and independently of each other. Find the probability that at least one of the $20$ selected people has influenza.\\\\
        When handling cases with 'Atleast one' it can be an advantage to look at the opposite:
		\[
		P(\text{Atleast one infected}) = 1 - P(\text{No infected})
		\]
		Since we are looking at independant trials when picking from the population, we can multiply the propability of each trials to get their intersected probability [1.4.1]\cite{STAT}.\\
		Let:
		\[
		P(\text{Is infected}) = P(B) = 0.1 \;\;\;\;\;\;\;\;\; P(\text{Not infected}) = P(B^c) = 0.9
		\]
		We are picking 20 people which is 20 independant trials hence our probability is:
		\[
		P(\text{No infected}) = 0.9^{20} = 0.121  
		\]
		Now we can look back at the initial proposition
		\[
		P(\text{Atleast one infected}) = 1-0.121 = \underbar{0.879} 
		\]
		We Did it!\\\\\\



     
	
    A person can be tested for influenza using a test, which does not always provide the correct answer.\\
	The test gives a positive result for influenza $85\%$ of the time for people who have influenza, and a negative result for influenza $95\%$ of the time for people who do not have influenza.  
    
    \item Find the probability of having influenza given that a person has tested positive for influenza.\\\\
	  We are looking for $P(B|A)$ where $B$ is the event that a person has influenza(Is infected) and $A$ is the event that a person tests positive.\\
	  For this we can use Bayes' Theorem [1.4.3]\cite{STAT}, that is used to invert conditional probabilities:
	  \[
	  P(B|A) = \frac{P(A|B)P(B)}{P(A)}
	  \]
	  But we don't know $P(A)$, but Bayes' Theorem can be manipulated to rewrite $P(A)$ [1.4.3]\cite{STAT}:
	  \[
	 P(B|A) = \frac{P(A|B)P(B)}{P(A|B)P(B) + P(A|B^c)P(B^c)}		 
	  \]
	  Now we can insert our known values:
	  \[
	  P(A|B) = 0.85 \;\;\;\;\;\; P(B) = 0.1 \;\;\;\;\;\; P(A|B^c) = 0.05 \;\;\;\;\;\; P(B^c) = 0.9
	  \]
	  And we get:
	  \[
	  	 P(B|A) = \frac{0.85 \cdot 0.1}{0.85 \cdot 0.1 + 0.05 \cdot 0.9} = \underbar{0.654}
	  \]
\end{enumerate}












\printbibliography % This prints the bibliography

\end{document}