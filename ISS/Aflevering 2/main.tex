%%%%%     PACKS     %%%%%
\documentclass[12pt]{article}
\usepackage[margin=1in,headsep=.60in]{geometry}
\usepackage[utf8]{inputenc}
\usepackage[table, dvipsnames]{xcolor}
\usepackage{array}
\usepackage{amsmath}
\usepackage{booktabs}

\usepackage{amssymb}
\usepackage{amsfonts}
\usepackage{siunitx}
\usepackage{graphicx}
\usepackage{caption}
\usepackage{pgfplots}
\graphicspath{{Images/}}
\usepackage[colorinlistoftodos]{todonotes}
\usepackage{cleveref}
\usepackage[labelformat=simple]{subcaption}
\usepackage{grffile}
\usepackage{gensymb}
\usepackage{float}
\usepackage[shortlabels]{enumitem}
\usepackage{enumitem}
\setlistdepth{9}
\usepackage{biblatex} %Imports biblatex package
\addbibresource{references.bib}
\usepackage{outlines}
\usepackage{minted}
\usepackage{pdflscape}
\usepackage{everypage}
\usepackage{multirow}
\usepackage{multicol}
\usepackage{afterpage}
\usepackage[labelfont=bf, font=small]{caption}
\usepackage{upgreek}
\usepackage{tikz}
\usetikzlibrary{automata, positioning}
\usepackage{logicproof}


\usepackage{lastpage}
%%%%%     COMMANDS     %%%%%
\newcommand{\p}{\partial}
\newenvironment{rowequmat}[1]{\left(\array{@{}#1@{}}}{\endarray\right)}


\begin{document}

\begin{titlepage}
\newcommand{\HRule}{\rule{\linewidth}{0.5mm}}
\center

\textsc{\LARGE Aarhus university}\\[1.5cm]
\textsc{\Large Computer-Science}\\[0.5cm]
\textsc{\large Introduction to probability and statistics}\\[0.5cm]
    

\HRule\\[0.4cm]
	
	{\huge\bfseries Handin 1 }\\[0.4cm] % Title of your document
	
\HRule\\[1.5cm]

\begin{minipage}{0.4\textwidth}
		\begin{flushleft}
			\large
			\textit{Author}\\
			Søren M. \textsc{Damsgaard}\\
                % Your name
		\end{flushleft}
	\end{minipage}
~
	\begin{minipage}{0.4\textwidth}
		\begin{flushright}
			\large
			\textit{Student number}\\
			\textbf{202309814}\\
                % Studienummer\
			
		\end{flushright}
	\end{minipage}

\vfill\vfill\vfill % Position the date 3/4 down the remaining page
	
	{\large\today}

\vfill\vfill
	\includegraphics[width=0.2\textwidth]{Aarhus_University_seal.png}\\[1cm] % Include a department/university logo - this will require the graphicx package
	 

\vfill
\end{titlepage}
%%%%%     CHAPTERS     %%%%%
\hspace{0.02cm}

\section*{Problem 28: In a factory there are 100 units of a certain product, 5 of which are defective. We pick
three units from the 100 units at random. What is the probability that exactly one of
them is defective?}
Let's look at the numbers we have.
\begin{align*}
& \text{Let our total number of units be } N = 100 \\
& \text{Let our defects be } b = 5 \\
& \text{Let the number of units we pick be } k = 3 \\
& \text{Let the number of defects we want be } x = 1 \\
& \text{Let our functional units be } r = 95
\end{align*}
This looks like a problem that can be solved with hypergeometric 
distribution, so
we will use the formula for hypergeometric distribution
given in definition [3.8]\cite{STAT}.\\
\[
\frac{\begin{pmatrix} b\\x\end{pmatrix}\begin{pmatrix} r\\k-x\end{pmatrix}
}{\begin{pmatrix} b+r\\k\end{pmatrix}}
\]
Filling in our numbers we get
\[
\frac{\begin{pmatrix} 5\\1\end{pmatrix}\begin{pmatrix} 95\\2\end{pmatrix}
}{\begin{pmatrix} 100\\3\end{pmatrix}}
\]
This problem can be seen as 'Unordered sampling without replacement' as
we are not putting the units back after picking them.\\
This means we use the binomial coefficient formula for unordered sampling without replacement
given in chapter [2.1.3]\cite{STAT}.\\
Please note that $n$ and $k$ are just variables and have no relation to the $n$ and $k$ we used earlier.
\[
\begin{pmatrix} n\\k\end{pmatrix} = \frac{n!}{k!(n-k)!}
\]
We have 3 binomial coefficients we have to calculate,  let's get to it!.
\[
\binom{5}{1} = \frac{5!}{1!4!} = \frac{120}{24} = 5
\]
For the next two we will let some of the factorials cancel out to make it easier.
\[
\binom{95}{2} = \frac{95!}{2!93!} = \frac{95 \cdot 94}{2} = 4465
\]
\[
\binom{100}{3} = \frac{100!}{3!97!} = \frac{100 \cdot 99 \cdot 98}{6} = 161700
\]
Putting it all together we get
\[
\frac{\begin{pmatrix} 5\\1\end{pmatrix}\begin{pmatrix} 95\\2\end{pmatrix}
}{\begin{pmatrix} 100\\3\end{pmatrix}} = \frac{5 \cdot 4465}{161700} = \frac{22325}{161700} \approx 0.138
\]
The probability of getting exactly one defective unit when picking 3 units is somethign along the lines of 13.8\%.\\
Obligatory victory cry! HUZZAR!












\printbibliography % This prints the bibliography

\end{document}