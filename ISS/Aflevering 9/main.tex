%%%%%     PACKS     %%%%%
\documentclass[12pt]{article}
\usepackage[margin=1in,headsep=.60in]{geometry}
\usepackage[utf8]{inputenc}
\usepackage[table, dvipsnames]{xcolor}
\usepackage{array}
\usepackage{amsmath}
\usepackage{booktabs}
\usepackage{mdframed} %For box around text

\usepackage{amssymb}
\usepackage{amsfonts}
\usepackage{siunitx}
\usepackage{graphicx}
\usepackage{caption}
\usepackage{pgfplots}
\graphicspath{{Images/}}
\usepackage[colorinlistoftodos]{todonotes}
\usepackage{cleveref}
\usepackage[labelformat=simple]{subcaption}
\usepackage{grffile}
\usepackage{gensymb}
\usepackage{float}
\usepackage[shortlabels]{enumitem}
\usepackage{enumitem}
\setlistdepth{9}
\usepackage{biblatex} %Imports biblatex package
\addbibresource{references.bib}
\usepackage{outlines}
\usepackage{minted}
\usepackage{pdflscape}
\usepackage{everypage}
\usepackage{multirow}
\usepackage{multicol}
\usepackage{afterpage}
\usepackage[labelfont=bf, font=small]{caption}
\usepackage{upgreek}
\usepackage{tikz}
\usetikzlibrary{automata, positioning}
\usepackage{logicproof}


\usepackage{lastpage}
%%%%%     COMMANDS     %%%%%
%%%% Blue box for subsection text (not figures) %%%%
\newenvironment{bluebox}
  {\begin{mdframed}[backgroundcolor=blue!5,linecolor=blue!40,roundcorner=8pt]}
  {\end{mdframed}}

%%%% Simple neutral box for figures %%%%
\newenvironment{figbox}
  {\begin{mdframed}[roundcorner=8pt,shadow=true,shadowsize=4pt,shadowcolor=black!40]}
  {\end{mdframed}}




\begin{document}

\begin{titlepage}
\newcommand{\HRule}{\rule{\linewidth}{0.5mm}}




\center

\textsc{\LARGE Aarhus university}\\[1.5cm]
\textsc{\Large Computer-Science}\\[0.5cm]
\textsc{\large Introduction to probability and statistics}\\[0.5cm]
    

\HRule\\[0.4cm]
	\center	
	{\huge\bfseries Handin 9 }\\[0.4cm] % Title of your document
\HRule\\[1.5cm]

\begin{minipage}{0.4\textwidth}
		\begin{flushleft}
			\large
			\textit{Author}\\	
			Søren M. \textsc{Damsgaard}\\
                % Your name
		\end{flushleft}
	\end{minipage}
~
	\begin{minipage}{0.4\textwidth}
		\begin{flushright}
			\large
			\textit{Student number}\\
			\textbf{202309814}\\
                % Studienummer\
			
		\end{flushright}
	\end{minipage}

\vfill\vfill\vfill % Position the date 3/4 down the remaining page
	
	{\large\today}

\vfill\vfill
	\includegraphics[width=0.2\textwidth]{Aarhus_University_seal.png}\\[1cm] % Include a department/university logo - this will require the graphicx package
	 

\vfill
\end{titlepage}
%%%%%     CHAPTERS     %%%%%
\hspace{0.02cm}
\begin{bluebox}
Let $X_1$ and $X_2$ denote two independent random variables such that 
$X_1 \sim \mathrm{Exponential}(\lambda_1)$ and 
$X_2 \sim \mathrm{Exponential}(\lambda_2)$, 
where $\lambda_1, \lambda_2 > 0$. 
Find the distribution of $Y := \min\{X_1, X_2\}$.
\end{bluebox}
We start by understanding the meaning of life...42.\\
And then we look at the CDF of $Y$, which by the definition of CDF is given by:
\[
F_Y(y) = P(Y \leq y) = P(\min\{X_1, X_2\} \leq y).
\]
this can also be written as:
\begin{equation}\label{start def of CDF Y}
P(\min\{X_1, X_2\} \leq y) = P(X_1 \leq y \cup X_2 \leq y).
\end{equation}
Which gives us the opportunity to use the inclusion-exclusion principle to calculate the CDF of $Y$ tha way:
\[
F_Y(y) = P(X_1 \leq y) + P(X_2 \leq y) - P(X_1 \leq y \cap X_2 \leq y).
\]
Since $X_1$ and $X_2$ are independent we can write:
\[
P(X_1 \leq y \cap X_2 \leq y) = P(X_1 \leq y)P( X_2 \leq y).
\]
which gives us:
\[
F_Y(y) = P(X_1 \leq y) + P(X_2 \leq y) - P(X_1 \leq y)P( X_2 \leq y).
\]
But that's a lot of work that i don't wanna do...
\texttt{Insert tired spongebob meme here}\\
Instead, we will do it the lazt way!\\\\	
Let's do some math magic on Equation \ref{start def of CDF Y} also know as taking the complement.
\[
 P(X_1 \leq y \cup X_2 \leq y) = 1 - P(X_1 > y \cap X_2 > y).
\]
Again, since $X_1$ and $X_2$ are independent we can write:
\[
 P(X_1 > y \cap X_2 > y) = P(X_1 > y)P(X_2 > y).
\]
Thus we have:
\begin{equation}\label{CDF Y complement}
F_Y(y) = 1 - P(X_1 > y)P(X_2 > y).
\end{equation}
We are smart and see that it looks like the memoryless property of the exponential distribution from chapter 4.2.2, which means we can use the expression:
\[
P(X > x) = e^{-\lambda x}
\]
We can then insert that into Equation \ref{CDF Y complement}:
\[
F_Y(y) = 1 - (e^{-\lambda y})(e^{-\lambda y}) = 1 - e^{-(\lambda_1 + \lambda_2)y}.
\]
Then to finish, we let $\lambda = \lambda_1 + \lambda_2$ and differentiate the CDF to get the PDF:
\begin{figbox}
\begin{figure}[H]
	\centering
	\includegraphics{image.png}
\end{figure}
\end{figbox}
Thus we see that $Y \sim \mathrm{Exponential}(\lambda_1 + \lambda_2)$.


\printbibliography% This prints the bibliography

\end{document}