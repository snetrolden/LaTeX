%%%%%     PACKS     %%%%%
\documentclass[12pt]{article}
\usepackage[margin=1in,headsep=.60in]{geometry}
\usepackage[utf8]{inputenc}
\usepackage[table, dvipsnames]{xcolor}
\usepackage{array}
\usepackage{amsmath}
\usepackage{booktabs}

\usepackage{amssymb}
\usepackage{amsfonts}
\usepackage{siunitx}
\usepackage{graphicx}
\usepackage{caption}
\usepackage{pgfplots}
\graphicspath{{Images/}}
\usepackage[colorinlistoftodos]{todonotes}
\usepackage{cleveref}
\usepackage[labelformat=simple]{subcaption}
\usepackage{grffile}
\usepackage{gensymb}
\usepackage{float}
\usepackage[shortlabels]{enumitem}
\usepackage{enumitem}
\setlistdepth{9}
\usepackage{biblatex} %Imports biblatex package
\addbibresource{references.bib}
\usepackage{outlines}
\usepackage{minted}
\usepackage{pdflscape}
\usepackage{everypage}
\usepackage{multirow}
\usepackage{multicol}
\usepackage{afterpage}
\usepackage[labelfont=bf, font=small]{caption}
\usepackage{upgreek}
\usepackage{tikz}
\usetikzlibrary{automata, positioning}
\usepackage{logicproof}


\usepackage{lastpage}
%%%%%     COMMANDS     %%%%%
\newcommand{\p}{\partial}
\newenvironment{rowequmat}[1]{\left(\array{@{}#1@{}}}{\endarray\right)}


\begin{document}

\begin{titlepage}
\newcommand{\HRule}{\rule{\linewidth}{0.5mm}}
\center

\textsc{\LARGE Aarhus university}\\[1.5cm]
\textsc{\Large Computer-Science}\\[0.5cm]
\textsc{\large Introduction to probability and statistics}\\[0.5cm]
    

\HRule\\[0.4cm]
	
	{\huge\bfseries Handin 1 }\\[0.4cm] % Title of your document
	
\HRule\\[1.5cm]

\begin{minipage}{0.4\textwidth}
		\begin{flushleft}
			\large
			\textit{Author}\\
			Søren M. \textsc{Damsgaard}\\
                % Your name
		\end{flushleft}
	\end{minipage}
~
	\begin{minipage}{0.4\textwidth}
		\begin{flushright}
			\large
			\textit{Student number}\\
			\textbf{202309814}\\
                % Studienummer\
			
		\end{flushright}
	\end{minipage}

\vfill\vfill\vfill % Position the date 3/4 down the remaining page
	
	{\large\today}

\vfill\vfill
	\includegraphics[width=0.2\textwidth]{Aarhus_University_seal.png}\\[1cm] % Include a department/university logo - this will require the graphicx package
	 

\vfill
\end{titlepage}
%%%%%     CHAPTERS     %%%%%
\hspace{0.02cm}

\section*{Problem 14}
Let A and B be two events such that
\[
P(A) = 0.4 \;\;\; P(B) = 0.7 \;\;\; P(A \cup B) = 0.9 
\]

\subsection*{Question (a) find $P(A \cap B)$}
We use the 'inclusion-exclusion-principle' 
\[
P(A\cap B) = P(A) + P(B) - P(A \cup B)
\]
\[
0.2 = 0.4 + 0.7 - 0.9
\]

\subsection*{Question (b) find $P(A^c \cap B)$}
$P(A^c \cap B)$ can be rewritten as $P(B) - P(A \cap B)$, this can be found from the equation in figure 1.16, which is values we know
\[
P(A^c \cap B) = P(B) - P(A \cap B)
 \]
 \[
 0.5 = 0.7 - 0.2
 \]


\subsection*{Question (c) find $P(A -B)$}
$P(A - B)$ is just another way to write $P(A \cap B^c)$, which is noted in figure 1.8 from the book, 
hence we get
\[
P(A-B) = P(A) - P(A\cap B)
\]
\[
0.2 = 0.4 - 0.2
\]

\subsection*{Question (d) find $P(A^c - B)$}
Again another way to write $P(A^c \cap B^c)$.\\
We can use De-morgans law
\[
P(A^c \cap B^c) = P((A\cup B)^c) 
\]
We can remove the 'complement' by subtracting 1
\[
P((A\cup B)^c) = 1- P(A \cup B)
\]
Now we can derive the probability since we already know the probability of $P(A \cup B)$
\[
0.1 = 1- 0.9
\]

\subsection*{Question (e) find $P(A^c \cup B)$}
We use the 'inclusion-exclusion-principle' again
\[
P(A^c \cup B) = P(A^c) + P(B) - P(A^c \cap B)
\]
\[
0.8 = 0.6 + 0.7 - 0.5
\]





\subsection*{Question (f) find $P(A \cap(B\cup A^c))$}
We use the distributive law
\[
P(A \cap(B\cup A^c)) = P((A\cap B) \cup (A\cap A^c))
\]
The set $(A\cap A^c)$ is the empty set $\phi$ so we are left with the set $(A \cap B)$, and we know the probability of this set from (a)
\[
P(A \cap(B\cup A^c)) = P(A \cap B) = 0.2
\]











\printbibliography % This prints the bibliography

\end{document}