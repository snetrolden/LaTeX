%%%%%     PACKS     %%%%%
\documentclass[12pt]{article}
\usepackage[margin=1in,headsep=.60in]{geometry}
\usepackage[utf8]{inputenc}
\usepackage[table, dvipsnames]{xcolor}
\usepackage{array}
\usepackage{amsmath}
\usepackage{booktabs}
\usepackage{mdframed} %For box around text

\usepackage{amssymb}
\usepackage{amsfonts}
\usepackage{siunitx}
\usepackage{graphicx}
\usepackage{caption}
\usepackage{pgfplots}
\graphicspath{{Images/}}
\usepackage[colorinlistoftodos]{todonotes}
\usepackage{cleveref}
\usepackage[labelformat=simple]{subcaption}
\usepackage{grffile}
\usepackage{gensymb}
\usepackage{float}
\usepackage[shortlabels]{enumitem}
\usepackage{enumitem}
\setlistdepth{9}
\usepackage{biblatex} %Imports biblatex package
\addbibresource{references.bib}
\usepackage{outlines}
\usepackage{minted}
\usepackage{pdflscape}
\usepackage{everypage}
\usepackage{multirow}
\usepackage{multicol}
\usepackage{afterpage}
\usepackage[labelfont=bf, font=small]{caption}
\usepackage{upgreek}
\usepackage{tikz}
\usetikzlibrary{automata, positioning}
\usepackage{logicproof}


\usepackage{lastpage}
%%%%%     COMMANDS     %%%%%
%%%% Blue box for subsection text (not figures) %%%%
\newenvironment{bluebox}
  {\begin{mdframed}[backgroundcolor=blue!5,linecolor=blue!40,roundcorner=8pt]}
  {\end{mdframed}}

%%%% Simple neutral box for figures %%%%
\newenvironment{figbox}
  {\begin{mdframed}[roundcorner=8pt,shadow=true,shadowsize=4pt,shadowcolor=black!40]}
  {\end{mdframed}}




\begin{document}

\begin{titlepage}
\newcommand{\HRule}{\rule{\linewidth}{0.5mm}}




\center

\textsc{\LARGE Aarhus university}\\[1.5cm]
\textsc{\Large Computer-Science}\\[0.5cm]
\textsc{\large Introduction to probability and statistics}\\[0.5cm]
    

\HRule\\[0.4cm]
	\center	
	{\huge\bfseries Handin 8 }\\[0.4cm] % Title of your document
\HRule\\[1.5cm]

\begin{minipage}{0.4\textwidth}
		\begin{flushleft}
			\large
			\textit{Author}\\	
			Søren M. \textsc{Damsgaard}\\
                % Your name
		\end{flushleft}
	\end{minipage}
~
	\begin{minipage}{0.4\textwidth}
		\begin{flushright}
			\large
			\textit{Student number}\\
			\textbf{202309814}\\
                % Studienummer\
			
		\end{flushright}
	\end{minipage}

\vfill\vfill\vfill % Position the date 3/4 down the remaining page
	
	{\large\today}

\vfill\vfill
	\includegraphics[width=0.2\textwidth]{Aarhus_University_seal.png}\\[1cm] % Include a department/university logo - this will require the graphicx package
	 

\vfill
\end{titlepage}
%%%%%     CHAPTERS     %%%%%
\hspace{0.02cm}
\begin{bluebox}
Consider the set of points in the set C:
\[
C = \{ (x,y)|x,y \in \mathbb{Z},x^2 + |y| \leq 2 \}
\]
Suppose that we pick a point $(X,Y)$ from this set completely at random. Thus, each
point has a probability $\frac{1}{11}$ of being chosen.
\end{bluebox}

\subsection*{Find the joint and marginal PMFs of $X$ and $Y$}
from the set C we can see that there are 11 points that satisfy the condition $x^2 + |y| \leq 2$:
\[
(0,0),(1,0),(-1,0),(0,1),(0,-1),(1,1),(1,-1),(-1,1),(-1,-1),(0,2),(0,-2)
\]
So the joint PMF is given by:
\[
P_{X,Y}(x,y) = \begin{cases}
\frac{1}{11} & (x,y) \in C\\
0 & \text{otherwise}
\end{cases}
\]
or it can also be written as a table like example 5.1~\cite{STAT}:
\[
\begin{array}{|c|c|c|c|c|c|}
\hline
P_{X,Y}(x,y) & y = -2 & y = -1 & y = 0 & y = 1 & y = 2 \\
\hline
x = -1 & 0 & \frac{1}{11} & \frac{1}{11} & \frac{1}{11} & 0 \\
\hline
x = 0 & \frac{1}{11} & \frac{1}{11} & \frac{1}{11} & \frac{1}{11} & \frac{1}{11} \\
\hline
x = 1 & 0 & \frac{1}{11} & \frac{1}{11} & \frac{1}{11} & 0 \\
\hline
\end{array}
\]
\noindent To find the marginal PMF of X we sum over all possible values of Y like in chapter 5 \textit{marginal PMFs}~\cite{STAT}:
\[
P_X(x) = \sum_{y_j \in \mathbb{R}_Y} P_{X,Y}(x,y_j)
\]
This gives us:
\[
P_X(x) = \begin{cases}
\frac{5}{11} & x = 0\\
\frac{3}{11} & x = 1\\
\frac{3}{11} & x = -1\\
0 & \text{otherwise}
\end{cases}
\]
Similarly, for the marginal PMF of Y we sum over all possible values of X:
\[
P_Y(y) = \sum_{x_i \in \mathbb{R}_X} P_{X,Y}(x_i,y)
\]	
This gives us:
\[
P_Y(y) = \begin{cases}
\frac{1}{11} & y = -2\\
\frac{3}{11} & y = -1\\
\frac{3}{11} & y = 0\\
\frac{3}{11} & y = 1\\	
\frac{1}{11} & y = 2\\
0 & \text{otherwise}
\end{cases}
\]
Note: Since we are in $\mathbb{Z}$ we can't include decimals when we draw C in a coordinate system, so the correct would only be to plot the points in C.
\begin{figbox}
\begin{figure}[H]
	\centering
	\includegraphics{image.png}
	\caption{The set C plotted in a coordinate system.}
\end{figure}
\end{figbox}

\subsection*{Find the conditional PMF of $X$ given $Y = 1$.}
The formula for finding the conditional PMF is given in chapter 5.1.3~\cite{STAT}:
\[
P_{X|Y}(x_i|y_i) = \frac{P_{XY}(x_i,y_j)}{P_Y(y_i)} 
\]
Then we sum all values of $X$ where $Y = 1$:
\[
\sum_{x_i \in R_X}^{} P_{X|Y}(x_i|1)
\]
This gets us:
\[
P_{X|Y}(-1|1) + P_{X|Y}(0|1) + P_{X|Y}(1,1)
\]
And we insert the numbers from the joint PMF:
\[
\frac{\frac{1}{11}}{\frac{3}{11}} + \frac{\frac{1}{11}}{\frac{3}{11}} + \frac{\frac{1}{11}}{\frac{3}{11}} = \frac{1}{3} + \frac{1}{3} + \frac{1}{3} = 1
\]
Which gives us the conditional PMF:
\[
P_{X|Y}(x|1) = \begin{cases}
\frac{1}{3} & x = -1\\
\frac{1}{3} & x = 0\\
\frac{1}{3} & x = 1\\
0 & \text{otherwise}
\end{cases}
\]
It all sums up to 1 which validates our findings of the conditional PMF according to axiom 2 of probability~\cite{STAT}.\\

\subsection*{Are $X$ and $Y$ independent?}
Here we take the time machine back to chapter 1.4.1~\cite{STAT} where independence was born. 
it is said in this chapter that when two variables are independant the following holds true:
\[P(A|B) = P(A)\]
This is applicable in our case, sinc we just found the conditional PMF of $X$ given $Y = 1$. So we just need to check if $P_{X|Y}(x|1) = P_X(x)$.\\
We can see that this is not the case since:
\[
P_{X|Y}(0|1) = \frac{1}{3} \neq P_X(0) = \frac{5}{11}
\]
hence $X$ and $Y$ are not independent. Womp Womp *Cry Emoji*.\\

\subsection*{Find $E[XY^2]$}
By the way of LOTUS and a little bit of equation 5.5~\cite{STAT} we can find $E[XY^2]$ like so:
\[
E[XY^2] = \sum_{(x_i,y_j) \in R_X} x_i y_j^2 \cdot P_{X,Y}(x_i,y_j)
\]
Before we insert the values we note that every point in C where $x = 0$ and $y = 0$ will result in $0$ when calculating $E[XY^2]$, so we exclude those points:
\[
E[XY^2] = 1\cdot1^2 \frac{1}{11} + (-1)\cdot1^2 \frac{1}{11} + 1\cdot(-1)^2 \frac{1}{11} + (-1)\cdot(-1)^2 \frac{1}{11}
\]
\[
E[XY^2] = \frac{1}{11} + \frac{1}{11} - \frac{1}{11} - \frac{1}{11} = 0
\]
\noindent Hence $E[XY^2] = 0$.

\printbibliography% This prints the bibliography

\end{document}