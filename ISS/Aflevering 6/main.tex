%%%%%     PACKS     %%%%%
\documentclass[12pt]{article}
\usepackage[margin=1in,headsep=.60in]{geometry}
\usepackage[utf8]{inputenc}
\usepackage[table, dvipsnames]{xcolor}
\usepackage{array}
\usepackage{amsmath}
\usepackage{booktabs}
\usepackage{mdframed} %For box around text

\usepackage{amssymb}
\usepackage{amsfonts}
\usepackage{siunitx}
\usepackage{graphicx}
\usepackage{caption}
\usepackage{pgfplots}
\graphicspath{{Images/}}
\usepackage[colorinlistoftodos]{todonotes}
\usepackage{cleveref}
\usepackage[labelformat=simple]{subcaption}
\usepackage{grffile}
\usepackage{gensymb}
\usepackage{float}
\usepackage[shortlabels]{enumitem}
\usepackage{enumitem}
\setlistdepth{9}
\usepackage{biblatex} %Imports biblatex package
\addbibresource{references.bib}
\usepackage{outlines}
\usepackage{minted}
\usepackage{pdflscape}
\usepackage{everypage}
\usepackage{multirow}
\usepackage{multicol}
\usepackage{afterpage}
\usepackage[labelfont=bf, font=small]{caption}
\usepackage{upgreek}
\usepackage{tikz}
\usetikzlibrary{automata, positioning}
\usepackage{logicproof}


\usepackage{lastpage}
%%%%%     COMMANDS     %%%%%
%%%% Blue box for subsection text (not figures) %%%%
\newenvironment{bluebox}
  {\begin{mdframed}[backgroundcolor=blue!5,linecolor=blue!40,roundcorner=8pt]}
  {\end{mdframed}}

%%%% Simple neutral box for figures %%%%
\newenvironment{figbox}
  {\begin{mdframed}[roundcorner=8pt,shadow=true,shadowsize=4pt,shadowcolor=black!40]}
  {\end{mdframed}}




\begin{document}

\begin{titlepage}
\newcommand{\HRule}{\rule{\linewidth}{0.5mm}}




\center

\textsc{\LARGE Aarhus university}\\[1.5cm]
\textsc{\Large Computer-Science}\\[0.5cm]
\textsc{\large Introduction to probability and statistics}\\[0.5cm]
    

\HRule\\[0.4cm]
\begin{mdframed}
	\center	
	{\huge\bfseries Handin 6 (Box Edition) }\\[0.4cm] % Title of your document
\end{mdframed}
\HRule\\[1.5cm]

\begin{minipage}{0.4\textwidth}
		\begin{flushleft}
			\large
			\textit{Author}\\	
			Søren M. \textsc{Damsgaard}\\
                % Your name
		\end{flushleft}
	\end{minipage}
~
	\begin{minipage}{0.4\textwidth}
		\begin{flushright}
			\large
			\textit{Student number}\\
			\textbf{202309814}\\
                % Studienummer\
			
		\end{flushright}
	\end{minipage}

\vfill\vfill\vfill % Position the date 3/4 down the remaining page
	
	{\large\today}

\vfill\vfill
	\includegraphics[width=0.2\textwidth]{Aarhus_University_seal.png}\\[1cm] % Include a department/university logo - this will require the graphicx package
	 

\vfill
\end{titlepage}
%%%%%     CHAPTERS     %%%%%
\hspace{0.02cm}
\begin{bluebox}
\subsection*{(a) Plot the probability mass function (PMF) of $X$, i.e., plot the function $P_X$.}
\end{bluebox}
This PMF is a result of the 'data-frame' created in the code section in Figure~\ref{fig:poisson pmf}\\
The PMF is also rounded down to three decimals.
\[
    P(X = x) =
    \begin{cases}
    0.049 & x = 0, \\[6pt]
    0.149 & x = 1, \\[6pt]
    0.224 & x = 2, \\[6pt]
    0.224 & x = 3, \\[6pt]
    0.168 & x = 4, \\[6pt]
    0.100 & x = 5, \\[6pt]
    0.050 & x = 6, \\[6pt]
    0.021 & x = 7, \\[6pt]
    0.008 & x = 8, \\[6pt]
    0.002 & x = 9, \\[6pt]
    0.001 & x = 10, \\[6pt]
    0 & \text{otherwise}.
    \end{cases}
\]
\begin{figbox}
\begin{figure}[H]
    \centering
    \includegraphics[width=\linewidth]{poissonMeme.png}
  \caption{PMF of $X \sim \text{Poisson}(\lambda)$ with $\lambda = 3$.}\label{fig:poisson pmf}
\end{figure}
\end{figbox}
\begin{bluebox}
\subsection*{(b) Simulate the mean (expected value) of $X$, and compare it with the theoretical result.}
\end{bluebox}
For a Poisson distributed random variable $X \sim \text{Poisson}(\lambda)$ the theoretical expected value is $\mathbb{E}[X] = \lambda$ according to example 3.13~\cite{STAT}.  
In our case $\lambda = 3$, so $\mathbb{E}[X] = 3$.  
\begin{figbox}
\begin{figure}[H]
  \centering
  \includegraphics[width=\linewidth]{simEX.png}
  \caption{Convergence of the simulated mean of $X$ compared with the theoretical mean ($\lambda=3$).}
  \label{fig:mean convergence}
\end{figure}
\end{figbox}
It's seen that for high enough number of samples, the simulated mean converges to the theoretical mean of 3, as shown in Figure~\ref{fig:mean convergence}.
\begin{bluebox}
\subsection*{(c) Simulate the mean of $\sqrt{X}$.}
\end{bluebox}
Since it's hard to simplify and calculate the theoretical $\mathbb{E}[\sqrt{X}]$ for a Poisson distribution, we will use Monty Carlo simulations to approximate it.
\begin{figbox}
\begin{figure}[H]
  \centering
  \includegraphics[width=\linewidth]{sqX.png}
  \caption{Simulated mean of $\sqrt{X}$ for $X \sim \text{Poisson}(3)$.}
  \label{fig:square root of X}
\end{figure}
\end{figbox}
\begin{bluebox}
\subsection*{(d) Simulate the standard deviation of $X$, and compare it with the theoretical result.}
\end{bluebox}
\begin{figbox}
\begin{figure}[H]
  \centering
  \includegraphics[width=\linewidth]{def_cat.png}
  \caption{Simulated standard deviation of $X \sim \text{Poisson}(3)$.}
  \label{fig:simulated Var X}
\end{figure}
\end{figbox}
According to problem 8 from chapter 3~\cite{STAT} we know for the Poisson distribution that $\mathrm{Var}(X) = \lambda$.\\
And for our $\lambda = 3$, we have $SD(X) = \sqrt{3} \approx 1.732$. Which compared to the simulated value in Figure~\ref{fig:simulated Var X} shows a pretty good convergence.

\printbibliography % This prints the bibliography

\end{document}