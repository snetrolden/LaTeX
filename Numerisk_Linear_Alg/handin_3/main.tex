%%%%%     PACKS     %%%%%
\documentclass[12pt]{article}
\usepackage[margin=1in,headsep=.60in]{geometry}
\usepackage[utf8]{inputenc}
\usepackage[table, dvipsnames]{xcolor}
\usepackage{array}
\usepackage{amsmath}
\usepackage{booktabs}
\usepackage{mdframed} %For box around text
\usepackage{hyperref} %For hyperlinks in the PDF
\hypersetup{
	colorlinks=true,
	linkcolor=blue,
	filecolor=magenta,
	urlcolor=cyan,
}
\urlstyle{same}
\usepackage{amsthm}
\usepackage{amssymb}
\usepackage{amsfonts}
\usepackage{siunitx}
\usepackage{graphicx}
\usepackage{caption}
\usepackage{pgfplots}
\graphicspath{{Images/}}
\usepackage[colorinlistoftodos]{todonotes}
\usepackage{cleveref}
\usepackage[labelformat=simple]{subcaption}
\usepackage{grffile}
\usepackage{gensymb}
\usepackage{float}
\usepackage[shortlabels]{enumitem}
\setlistdepth{9}
\usepackage{biblatex} %Imports biblatex package
\addbibresource{references.bib}
\usepackage{outlines}
\usepackage{minted}
\usepackage{pdflscape}
\usepackage{everypage}
\usepackage{multirow}
\usepackage{multicol}
\usepackage{afterpage}
\usepackage[labelfont=bf, font=small]{caption}
\usepackage{upgreek}
\usepackage{tikz}
\usepackage{listings}

% style sheet for code
\definecolor{codegreen}{rgb}{0,0.6,0}
\definecolor{codegray}{rgb}{0.5,0.5,0.5}
\definecolor{codepurple}{rgb}{0.58,0,0.82}
\definecolor{backcolour}{rgb}{0.95,0.95,0.92}

\lstdefinestyle{mystyle}{
    backgroundcolor=\color{backcolour},   
    commentstyle=\color{codegreen},
    keywordstyle=\color{magenta},
    numberstyle=\tiny\color{codegray},
    stringstyle=\color{codepurple},
    basicstyle=\ttfamily\footnotesize,
    breakatwhitespace=false,         
    breaklines=true,                 
    captionpos=b,                    
    keepspaces=true,                 
    numbers=left,                    
    numbersep=5pt,                  
    showspaces=false,                
    showstringspaces=false,
    showtabs=false,                  
    tabsize=2
}

\lstset{style=mystyle}

% For pmatrix and augmented matrices
\makeatletter
\renewcommand*\env@matrix[1][*\c@MaxMatrixCols c]{%
  \hskip -\arraycolsep
  \let\@ifnextchar\new@ifnextchar
  \array{#1}}
\makeatother






\usepackage{lastpage}
%%%%%     COMMANDS     %%%%%
%%%% Blue box for subsection text (not figures) %%%%
\newenvironment{bluebox}
  {\begin{mdframed}[backgroundcolor=blue!5,linecolor=blue!40,roundcorner=8pt]}
  {\end{mdframed}}

%%%% Simple neutral box for figures %%%%
\newenvironment{figbox}
  {\begin{mdframed}[roundcorner=8pt,shadow=true,shadowsize=4pt,shadowcolor=black!40]}
  {\end{mdframed}}




\begin{document}

\begin{titlepage}
\newcommand{\HRule}{\rule{\linewidth}{0.5mm}}

\center

\textsc{\LARGE Aarhus university}\\[1.5cm]
\textsc{\Large Computer-Science}\\[0.5cm]
\textsc{\large Numerical Linear Algebra}\\[0.5cm]
    

\HRule\\[0.4cm]
    \center 
    {\huge\bfseries Handin 3 }\\[0.4cm] % Title of your document
\HRule\\[1.5cm]

\begin{minipage}{0.4\textwidth}
        \begin{flushleft}
            \large
            \textit{Author}\\   
            Søren M. \textsc{Damsgaard}\\
                 % Your name
        \end{flushleft}
    \end{minipage}
~
    \begin{minipage}{0.4\textwidth}
        \begin{flushright}
            \large
            \textit{Student number}\\
            \textbf{202309814}\\
                 % Studienummer\
            
        \end{flushright}
    \end{minipage}

\vfill\vfill\vfill % Position the date 3/4 down the remaining page
    
    {\large\today}

\vfill\vfill
    \includegraphics[width=0.2\textwidth]{Aarhus_University_seal.png}\\[1cm] % Include a department/university logo - this will require the graphicx package
    

\vfill
\end{titlepage}
%%%%%      CHAPTERS      %%%%%
\hspace{0.02cm}

\section*{Polynomials for days!}
As a postface, epilogue or early sequel to this piece of paper, the Author has become increasingly aware that making things pretty witty unicorn glitty, while keeping a formal tone, is consuming a lot of time, and is contrived for any lack of formality the reader might encounter. The Author urges the reader to use the following link if any discomfort arises in reading the paper. \href{https://www.youtube.com/watch?v=5wb5HWVh6Fs&list=RD5wb5HWVh6Fs&start_radio=1}{Emergency Witty Unicorn Hotline}
 
\subsection*{(a) Plotting like a Caveman}
Author goes plot plot with python
\begin{figure}[H]
    \centering
    \includegraphics[width=0.8\textwidth]{Figure_1.png}
    \caption{Plot of the data points}
    \label{fig:a}
\end{figure}
The code for the plot can be found in the appendix, \textsc{Line 5-7 and 70-75}
\subsection*{(b) System of equations, but make it polynomial}
Setting up system of equation for $p_1$, $p_2$ and $p_3$, going through the last three data points:
\begin{align*}
    &p_1(9) = a_1 + b_1 \cdot 9 + c_1 \cdot 9^2 = 45 \\
    &p_1(11) = a_1 + b_1 \cdot 11 + c_1 \cdot 11^2 = 65 \\
    &p_1(12) = a_1 + b_1 \cdot 12 + c_1 \cdot 12^2 = 70
\end{align*}
This give the following augmented matrix
\[
\begin{pmatrix}
\begin{array}{ccc|c}
1 & 9 & 81 & 45 \\
1 & 11 & 121 & 65 \\
1 & 12 & 144 & 70 \\
\end{array}
\end{pmatrix}
\]
Now to reduce it to something useful, by doing the following operations:
\begin{align*}
    &R_2 \to R_2 - R_1 \\
    &R_3 \to R_3 - R_1 \\
    &R_2 \to \frac{1}{2} R_2 \\
    &R_3 \to R_3 - 3R_2
\end{align*}
This gives us the following matrix:
\[
\begin{pmatrix}
\begin{array}{ccc|c}
1 & 9 & 81 & 45 \\
0 & 1 & 20 & 10 \\
0 & 0 & 3 & -5 \\
\end{array}
\end{pmatrix}
\]
With this we can easily find $a_1$, $b_1$ and $c_1$ by back substitution, which gives us $a_1 = -210$, $b_1 = 43.3$ and $c_1 = -\frac{5}{3}$. The code for this can be found in the appendix, \textsc{Line 10-18}.\\
And then we plot!
\begin{figure}[H]
    \centering
    \includegraphics[width=0.8\textwidth]{Figure_2.png}
    \caption{Plot of the polynomial running through the last three data points}
    \label{fig:b}
\end{figure}
the code for the plot can be found in the appendix, \textsc{Line 19-21 and 76-83}



\subsection*{(c) Vandermonde, Vandermonde, Vandermonde}
Okay here the Author gets a bit unsure, since the use of the Vandermonde matrix is used but isn't explained throughly until chapter 16 in the notes, which hasn't been covered yet.\\
The Vandermonde matrix essentially takes the x values and creates a matrix where each row $i$ is the x values raised to the power of $i$, augment this with the y values and solve for the coefficients via \textsc{np.vander()} and \textsc{np.linalg.solve()} respectively, see Appendix \textsc{Line 25-36}.\\
This should give the y values that allows the plotting of the polynomial of 4 degrees through all points, which is shown below. 
\begin{figure}[H]
    \centering
    \includegraphics[width=0.8\textwidth]{Figure_3.png}
    \caption{Plot of the polynomial of 4 degrees through all points}
    \label{fig:c}
\end{figure}
For plotting see Appendix \textsc{Line 84-91}

\subsection*{(d) Too many polynumials, not enough time}
Setting up the system of equations for a new set of $p_1$ and $p_2$:

\begin{align*}
    &p_1(2) = a_1 + b_1 \cdot 2 + c_1 \cdot 2^2 + d_1 \cdot 2^3 = 25 \\
    &p_1(6) = a_1 + b_1 \cdot 6 + c_1 \cdot 6^2 + d_1 \cdot 6^3 = 35 \\
    &p_1(9) = a_1 + b_1 \cdot 9 + c_1 \cdot 9^2 + d_1 \cdot 9^3 = 45 
\end{align*}

\begin{align*}
    &p_2(9) = a_2 + b_2 \cdot 9 + c_2 \cdot 9^2 + d_2 \cdot 9^3 = 45 \\
    &p_2(11) = a_2 + b_2 \cdot 11 + c_2 \cdot 11^2 + d_2 \cdot 11^3 = 65 \\
    &p_2(12) = a_2 + b_2 \cdot 12 + c_2 \cdot 12^2 + d_2 \cdot 12^3 = 70 
\end{align*}

To find where $p_1'(9) = p_2'(9)$, find the derivates then set them equal to each other and then group them on the left side.
\begin{align*}
    &p_1'(9) = b_1 + 2c_1 \cdot 9 + 3d_1 \cdot 9^2 \\
    &p_2'(9) = b_2 + 2c_2 \cdot 9 + 3d_2 \cdot 9^2 \\
    &b_1 + 2c_1 \cdot 9 + 3d_1 \cdot 9^2 = b_2 + 2c_2 \cdot 9 + 3d_2 \cdot 9^2\\
    &b_1 + 2c_1 \cdot 9 + 3d_1 \cdot 9^2 - b_2 - 2c_2 \cdot 9 - 3d_2 \cdot 9^2 = 0
\end{align*}
Then the system can be erected for $[a_1, b_1, c_1, d_1, a_2, b_2, c_2, d_2]$
\[
\begin{pmatrix}
\begin{array}{cccccccc|c}
1 & 2 & 4 & 8 & 0 & 0 & 0 & 0 & 25 \\
1 & 6 & 36 & 216 & 0 & 0 & 0 & 0 & 35 \\
1 & 9 & 81 & 729 & 0 & 0 & 0 & 0 & 45 \\
0 & 0 & 0 & 0 & 1 & 9 & 81 & 729 & 45 \\
0 & 0 & 0 & 0 & 1 & 11 & 121 & 1331 & 65 \\
0 & 0 & 0 & 0 & 1 & 12 & 144 & 1728 & 70 \\
0 & 1 & 18 & 243 & 0 & -1 & -18 & -243 & 0 \\
\end{array} 
\end{pmatrix} 
\]


\subsection*{(e) Sherlock Holmes and the One Solution}
The above systems in (d) is consistent, since no equation contradicts another. Altough it is not unique since there are 7 equations with 8 unknowns, which means the system is underdetermined and has infinitely many solutions.\\
The Author does not want to argue for consistency, since it makes the Author sad.
\subsection*{(f) Math Tariffs}
The assignment recommends imposing the condition $p_1'(0.0) = 0$, which will be used.\\
This conditions implies the slope at $t = 0$ is 0, which means the drone starts with no heat increase, which is a resonable assumption for a drone that hasn't started flying yet.\\
If we look at the derivative of $p_1$ and set $x = 0$ we get:
\begin{equation}
    p_1'(0) = b_1 + 2c_1 \cdot 0 + 3d_1 \cdot 0^2 = b_1
\end{equation}
This means that $b_1$ must be 0, which reduces the number of unknowns from 8 to 7. This means that the system of equations now have and equal number of equations and unknowns making it unique, according to chapter 6.4 in the notes. Note that consistency of the system was argued for in the previous section.
The resulting augmented matrix is the same as the one in (d) but with an extra row for the new conditon, see Appendix \textsc{Line 44-56} for the code:
\[
\begin{pmatrix}
\begin{array}{cccccccc|c}
1 & 2 & 4 & 8 & 0 & 0 & 0 & 0 & 25 \\
1 & 6 & 36 & 216 & 0 & 0 & 0 & 0 & 35 \\
1 & 9 & 81 & 729 & 0 & 0 & 0 & 0 & 45 \\
0 & 0 & 0 & 0 & 1 & 9 & 81 & 729 & 45 \\
0 & 0 & 0 & 0 & 1 & 11 & 121 & 1331 & 65 \\
0 & 0 & 0 & 0 & 1 & 12 & 144 & 1728 & 70 \\
0 & 1 & 18 & 243 & 0 & -1 & -18 & -243 & 0 \\
0 & 1 & 0 & 0 & 0 & 0 & 0 & 0 & 0 \\
\end{array} 
\end{pmatrix} 
\]
With this it is now possible to solve it and plot it, see Appendix \textsc{Line 57-66 and 94-102} for the code and plotting of Figure 4 respectively.
\begin{figure}[H]
    \centering
    \includegraphics[width=0.8\textwidth]{Figure_4.png}
    \caption{Plot of the spliced polynomials with the new condition}
    \label{fig:d}
\end{figure}
\subsection*{(g) Judicial Bias}





%put code in the appendix and refer to it here
\section*{Appendix}
\begin{lstlisting}[language=Python, caption=Python code for handin 3, label=lst:code]
import numpy as np
import matplotlib.pyplot as plt


# ...................a
time = np.array([2.0, 6.0, 9.0, 11.0, 12.0])
temp = np.array([25.0, 35.0, 45.0, 65.0, 70.0])

# -------------------b 
# System of equations
A = np.array([[1,9,81, 45],
                [1,11,121, 65],
                [1,12,144, 70]])
# Row operations
A[1,:] -= A[0,:]      # R_2 -> R_2 - R_1
A[2,:] -= A[0,:]      # R_3 -> R_3 - R_1
A[1,:] = (1/2)*A[1,:] # R_2 -> 1/2*R_2
A[2,:] -= 3*A[1,:]    # R_3 -> R_3 - 3R_2
tB = np.linspace(9.0, 12.0, 100)

y_valsB = (-210)+(130/3)*tB -(5/3)*tB**2 # 't' is the x values



# ----------c - polynomial interpolation

a = np.vander(time, len(time)) # The lenght is equal to 5 which corresponds to the exercise, which gives us 4 degrees (0,1,2,3,4)

# Augment and solve, This gives us the coefficients [a,b,c] see page 214 in notes
coefficientsC = np.linalg.solve(a, temp)

# give it some more 'doug' to work with
tC = np.linspace(2.0, 12.0, 100)

# This gives us p(x)
y_valsC = np.vander(tC, len(time)) @ coefficientsC



# -----------d One solution to rule them all
# setup a a system of equaitons for both p_1 and p_2 and for checking the slope, take the derivative of them both,
#subtract the derivates and set them to 0

# -----------f ALAKAZAM!
# time to construct big ass, bitch ass, matrix...fml
iCri_X = np.zeros((8,8))
iCri_Res = np.array([25,35,45,45,65,70,0,0])

iCri_X[0,:] =[1,2,4,8,0,0,0,0]
iCri_X[1,:] =[1,6,36,216,0,0,0,0]
iCri_X[2,:] =[1,9,81,729,0,0,0,0]
iCri_X[3,:] =[0,0,0,0,1,9,81,729]
iCri_X[4,:] =[0,0,0,0,1,11,121,1331]
iCri_X[5,:] =[0,0,0,0,1,12,144,1728]
iCri_X[6,:] =[0,1,18,243,0,-1,-18,-243]
iCri_X[7,1] =1 # The constraint b_1 = 0
#No need to vander before solve since our shiz is already a polynomial matrix
coefficients = np.linalg.solve(iCri_X, iCri_Res)

# splitting since we need to differenciate between p_1 and p_2 
t1 = np.linspace(2,9,50) # p_1
t2 = np.linspace(9,12,50) # p_2


y1 = np.vander(t1, 4, increasing=True) @ coefficients[:4]
y2 = np.vander(t2, 4, increasing=True) @ coefficients[4:]


#-------------PLotting-------------#
# Exercise a
figA, drone = plt.subplots()
drone.plot(time, temp, 'o-')
drone.set_xlabel('Time(min)')
drone.set_ylabel('Temepratur(C)')
drone.set_title('Temeperatur of drone over time #Ugly')
# Exercise b
figb, b = plt.subplots()
b.plot(time, temp, 'o', label = 'Data pointies' )
b.plot(tB,y_valsB, label = 'Polynomial running through the last 3 points' ) 
b.set_xlabel('Time(min)')
b.set_ylabel('Temepratur(C)')
b.set_title('Polynomial passing through the last 3 data points')
b.legend()
# Exercise c
figB, vander = plt.subplots()
vander.plot(time, temp, 'o', label = 'Data pointies')
vander.plot(tC, y_valsC, label = 'Polynomial of 4 degrees through all points' )
vander.set_xlabel('Time(min)')
vander.set_ylabel('Temepratur(C)')
vander.set_title('High Polynomials')
vander.legend()


# Exercise f
figC, big = plt.subplots()
big.plot(t1, y1, label='p1(x)')
big.plot(t2, y2, label='p2(x)')
big.plot(time, temp, 'o', label = 'Data pointies')
big.set_xlabel('Time(min)')
big.set_ylabel('Temepratur(C)')
big.set_title('Splicing polynomials and fitting them together ')
big.legend()



plt.show()
\end{lstlisting}
\end{document}
