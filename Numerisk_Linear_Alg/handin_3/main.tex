%%%%%     PACKS     %%%%%
\documentclass[12pt]{article}
\usepackage[margin=1in,headsep=.60in]{geometry}
\usepackage[utf8]{inputenc}
\usepackage[table, dvipsnames]{xcolor}
\usepackage{array}
\usepackage{amsmath}
\usepackage{booktabs}
\usepackage{mdframed} %For box around text
\usepackage{hyperref} %For hyperlinks in the PDF
\hypersetup{
	colorlinks=true,
	linkcolor=blue,
	filecolor=magenta,
	urlcolor=cyan,
}
\urlstyle{same}
\usepackage{amsthm}
\usepackage{amssymb}
\usepackage{amsfonts}
\usepackage{siunitx}
\usepackage{graphicx}
\usepackage{caption}
\usepackage{pgfplots}
\graphicspath{{Images/}}
\usepackage[colorinlistoftodos]{todonotes}
\usepackage{cleveref}
\usepackage[labelformat=simple]{subcaption}
\usepackage{grffile}
\usepackage{gensymb}
\usepackage{float}
\usepackage[shortlabels]{enumitem}
\usepackage{enumitem}
\setlistdepth{9}
\usepackage{biblatex} %Imports biblatex package
\addbibresource{references.bib}
\usepackage{outlines}
\usepackage{minted}
\usepackage{pdflscape}
\usepackage{everypage}
\usepackage{multirow}
\usepackage{multicol}
\usepackage{afterpage}
\usepackage[labelfont=bf, font=small]{caption}
\usepackage{upgreek}
\usepackage{tikz}
\usepackage{xcolor}
\usepackage{listings}

% style sheet for code
\definecolor{codegreen}{rgb}{0,0.6,0}
\definecolor{codegray}{rgb}{0.5,0.5,0.5}
\definecolor{codepurple}{rgb}{0.58,0,0.82}
\definecolor{backcolour}{rgb}{0.95,0.95,0.92}

\lstdefinestyle{mystyle}{
    backgroundcolor=\color{backcolour},   
    commentstyle=\color{codegreen},
    keywordstyle=\color{magenta},
    numberstyle=\tiny\color{codegray},
    stringstyle=\color{codepurple},
    basicstyle=\ttfamily\footnotesize,
    breakatwhitespace=false,         
    breaklines=true,                 
    captionpos=b,                    
    keepspaces=true,                 
    numbers=left,                    
    numbersep=5pt,                  
    showspaces=false,                
    showstringspaces=false,
    showtabs=false,                  
    tabsize=2
}

\lstset{style=mystyle}






\usepackage{lastpage}
%%%%%     COMMANDS     %%%%%
%%%% Blue box for subsection text (not figures) %%%%
\newenvironment{bluebox}
  {\begin{mdframed}[backgroundcolor=blue!5,linecolor=blue!40,roundcorner=8pt]}
  {\end{mdframed}}

%%%% Simple neutral box for figures %%%%
\newenvironment{figbox}
  {\begin{mdframed}[roundcorner=8pt,shadow=true,shadowsize=4pt,shadowcolor=black!40]}
  {\end{mdframed}}




\begin{document}

\begin{titlepage}
\newcommand{\HRule}{\rule{\linewidth}{0.5mm}}

\center

\textsc{\LARGE Aarhus university}\\[1.5cm]
\textsc{\Large Computer-Science}\\[0.5cm]
\textsc{\large Numerical Linear Algebra}\\[0.5cm]
    

\HRule\\[0.4cm]
    \center 
    {\huge\bfseries Handin 3 }\\[0.4cm] % Title of your document
\HRule\\[1.5cm]

\begin{minipage}{0.4\textwidth}
        \begin{flushleft}
            \large
            \textit{Author}\\   
            Søren M. \textsc{Damsgaard}\\
                 % Your name
        \end{flushleft}
    \end{minipage}
~
    \begin{minipage}{0.4\textwidth}
        \begin{flushright}
            \large
            \textit{Student number}\\
            \textbf{202309814}\\
                 % Studienummer\
            
        \end{flushright}
    \end{minipage}

\vfill\vfill\vfill % Position the date 3/4 down the remaining page
    
    {\large\today}

\vfill\vfill
    \includegraphics[width=0.2\textwidth]{Aarhus_University_seal.png}\\[1cm] % Include a department/university logo - this will require the graphicx package
    

\vfill
\end{titlepage}
%%%%%      CHAPTERS      %%%%%
\hspace{0.02cm}

\section*{Polynomials for days!}
 
\subsection*{(a) Plotting like a Caveman}

\subsection*{(b) System of equations, but make it polynomial}

\subsection*{(c) Vandermonde, Vandermonde, Vandermonde}

\subsection*{(d) Too many polynumials, not enough time}

\subsection*{(e) Sherlock Holmes and the One Solution}

\subsection*{(f) Math Tariffs}

\subsection*{(g) Judicial Bias}





%put code in the appendix and refer to it here
\section*{Appendix}
\begin{lstlisting}[language=Python, caption=Python code for handin 3, label=lst:code]
import numpy as np
import matplotlib.pyplot as plt


# a
time = np.array([2.0, 6.0, 9.0, 11.0, 12.0])
temp = np.array([25.0, 35.0, 45.0, 65.0, 70.0])

figA, drone = plt.subplots()
drone.plot(time, temp, 'o-')
drone.set_xlabel('Time(min)')
drone.set_ylabel('Temepratur(◦C)')
drone.set_title('Temeperatur of drone under operation #Ugly')



# b 
# Do this part on paper

# c - polynomial interpolation

a = np.vander(time, len(time)) # The lenght is equal to 5 which corresponds to the exercise, which gives us 4 degrees (0,1,2,3,4)

# Augment and solve, This gives us the coefficients [a,b,c] see page 214 in notes
coes = np.linalg.solve(a, temp)

t = np.linspace(2.0, 12.0, 100)

# This gives us p(x)
y_vals = np.vander(t, len(time)) @ coes


figC, vander = plt.subplots()
vander.plot(time, temp, 'o', label = 'Data pointies')
vander.plot(t, y_vals, label = '4 Degrees of Poly' )
vander.set_xlabel('Time(min)')
vander.set_ylabel('Temepratur(◦C)')



# d
# setup a a system of equaitons for both p_1 and p_2 and for checking the slope, take the derivative of them both,
#subtract the derivates and set them to 0


plt.show()
\end{lstlisting}
\printbibliography
\end{document}
